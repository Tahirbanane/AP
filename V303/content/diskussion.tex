\newpage
\section{Diskussion}
\label{sec:Diskussion}
Während der Durchführung des Versuchs sind einige Fehlerquellen aufgefallen, die die Ergebnisse beeinflussen.

\noindent
Bei der Untersuchung des Lock-In-Verstärkers treten Messunsicherheiten beim Ablesen vom Oszilloskop auf.

\noindent
Des Weiteren sind Schwankungen der Spannungen durch Wackelkontakte an den Kabeln aufgetreten.

\noindent
Diese Faktoren könnten Gründe für die große Abweichung einiger Messwerte von der Fitfunktion aus Abbildung (\ref{fig:phase}) sein.

\noindent
Allgemein wurden jedoch cosinus-Funktionen für die Messwerte gefunden, sodass der Zusammenhang aus Formel (\ref{eqn:uout2}) verifiziert wurde.

\noindent
Bei der Untersuchung der Photodiode gelten auch die Messunsicherheiten beim Ablesen vom Oszilloskop auf.

\noindent
Außerdem sind kleine Ablesefehler von der Skala bei der Bestimmung des Abstands $d$ sind denkbar.

\noindent
Durch den Fit aus Abbildung (\ref{fig:led}) kann eine 1/r Abhängigkeit der Intensität der Photodiode erkannt werden.
