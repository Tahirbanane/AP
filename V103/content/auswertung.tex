\newpage
\section{Auswertung}

In diesem Abschnitt wird der Versuch ausgewertet.

\subsection{Berechnung der Flächenträgheitsmomente}
Zunächst müssen die Flächenträgheitsmomente der Stäbe berechnet werden. Es liegen drei Arten von Stäben vor. Zwei mit runder und jeweils einer mit rechteckiger und  
 quadratischer Querschnittsfläche.

\noindent
Für den runden Stab gilt, mit $d$ als den Durchmesser,
\begin{align}
    \symbf{I}_\text{R} &= \int_0^{2\pi} \int_0^\frac{d}{2} (r \cdot \sin(\varphi))²\cdot \symup{d}r \symup{d}\varphi
    = \frac{1}{4}\left(\frac{d}{2}\right)⁴ \int_0^{2\pi} \sin(\varphi)² \symup{d}\varphi\\
    &= \left.\frac{d⁴}{64} \frac{\varphi}{2} - \frac{\sin(\varphi) \cos{\varphi}}{2} \right|_0^2\pi
    = \frac{\pi}{64} d⁴.
    \label{eqn:Rund}
\end{align}

\noindent
Für einen Stab mit rechteckiger Querschnittsfläche folgt dann äquivalent mit den Seitenlängen $a$ und $b$
\begin{equation}
    \symbf{I}_\text{Recht} = \int_\frac{-a}{2}^\frac{a}{2} \int_\frac{-b}{2}^\frac{b}{2} y² \symup{d}y \symup{d}z 
    = \int_\frac{-b}{2}^\frac{b}{2} \frac{a³}{12} \symup{d}z 
    = b \cdot \frac{a³}{12} 
    = \frac{a⁴}{12}
    \label{eqn:Quadratisch}
\end{equation}

\noindent
und speziell für eine quadratische Flächen
\begin{equation}
    \symbf{I}_\text{Q} = \frac{a⁴}{12}.
    \label{eqn:Quadratisch}
\end{equation}

\subsection{Bestimmung der Metalle}
Um die Metalle später vergleichen zu können ist es hilfreich über die Dichte diese zu bestimmen. Dies kann mit dem Zusammenhang
\begin{equation}
    \rho = \frac{m}{V}
    \label{rqn:dichte}
\end{equation}

\noindent
erfolgen. Die Stäbe werden gewogen und abgemessen dem Durchmesser $d$, der Masse $m$ und der Länge $l$, sowie den Seitenlängen $a$ und $b$

\begin{table}
	\centering
	\caption{Messwerte zu den Rundenstäben.} 
	\label{tab:vana} 
	\begin{tabular}{c c c}
	\toprule
	$d \, / \, \si{\centi\meter}$ & $m \, / \, \si{\gram} $ & $l \, / \, \si{\centi\meter}$\\
	\midrule
    1   &   393.8   &   60.0 \\
    1   &   416.6   &   60.1 \\
\bottomrule
	\end{tabular}
\end{table}

\begin{table}
	\centering
	\caption{Messwerte zu den eckigen Messtäben.} 
	\label{tab:vana} 
	\begin{tabular}{c c c c}
	\toprule
	$a \, / \, \si{\centi\meter}$ & $b \, / \, \si{\centi\meter}$ & $m \, / \, \si{\gram} $ & $l \, / \, \si{\centi\meter}$\\
	\midrule
    1   &   1   &   163.4   &   59.1 \\
    1   &   1.2   &  603.8  &   60.4 \\
\bottomrule
	\end{tabular}
\end{table}

\noindent
Das Volumen der runden Stäbe ist gegeben durch $V_\text{r} = \pi \cdot \left( \frac{d}{2} \right)^2 \cdot l$ und das Volumen der eckigen Stäbe durch 
$V_\text{e} = a \cdot b \cdot l$.

\noindent
Daraus folgt für das die Dichten gegeben sind durch
\begin{align*}
    m = 393.8 \, \si{\gram} &: \rho = 8.356\, \si{\gram\per\centi\meter\tothe{3}}\\
    m = 416.6 \, \si{\gram} &: \rho = 8.840\, \si{\gram\per\centi\meter\tothe{3}}\\
    m = 163.4 \, \si{\gram} &: \rho = 2.764\, \si{\gram\per\centi\meter\tothe{3}}\\
    m = 603.8 \, \si{\gram} &: \rho = 8.330\, \si{\gram\per\centi\meter\tothe{3}}\\
\end{align*}    

\noindent
Mit der Dichte lässt sich die Stange mit einer Masse von $m = 393.8 \, \si{\gram}$ und die mit einer Masse von $m = 603.8 \, \si{\gram}$ als Messing (\cite{Messing}), 
die Stange mit einer Masse von $m = 416.6 \, \si{\gram}$ als Kupfer (\cite{Kupfer}) und
die Stange mit einer Masse von $m = 163.4 \, \si{\gram}$ als Aluminium (\cite{Aluminium}) identifizieren. Diese Identifikationen passen auch mit den beobachteten Farben
der Stange.

\subsection{Einseitig eingehängt Stange}






