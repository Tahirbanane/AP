\newpage
\section{Diskussion}
\label{sec:Diskussion}

Die Elastizitätsmodule der Literatur (\cite{Elastizitätsmodul}) sind gegeben durch
\begin{align*}
     E_\text{Messing,lit} &= (0.78 - 1.23) \cdot 10^{11}  \, \si{\newton\per\meter²}\\
     E_\text{Kupfer,lit} &= 1.20 \cdot 10^{11} \, \si{\newton\per\meter²}\\
     E_\text{Aluminium,lit} &= 0.7 \cdot 10^{11} \, \si{\newton\per\meter²} \, . \\
\end{align*} 

Dardurch das Messing eine Legierung ist gibt es kein eideutiges Elastizitätsmodul, daher wird das um es leichter vergleichen zu können gemittelt zu $1.005 \cdot 10^{11} \, \si{\newton\per\meter²}$.

\noindent
Der relative Fehler lässt sich im allgemeinen berechnen mit 
\begin{equation*}
    \Delta p = \bigg |\frac{f-g}{f} \bigg | \, .
\end{equation*}

Es fällt auf, dass das Elastizitätsmodul, welches mittels der doppelteingespannten Stäbe, um mindestens den Faktor 2 bis 3 größer ist als das Elastizitätsmodul, welches
durch das einseitige einhängen bestimmt wurde. Daher wird die Angabe des relativen Fehlers verzichtet, da dieser sehr groß ist.
Die Fehler des einseitigen einhängens lässt sich bestimmen zu
\begin{align*}
    p_\text{Messung,ms} =& 46.07 \si{\percent}   \\
    p_\text{Messung,mr} =& 9.75 \si{\percent}  \\
    p_\text{Messung,as} =& 11.85 \si{\percent}\\
    p_\text{Messung,cr} =& 17.5 \si{\percent} \, .
\end{align*}

\noindent
Gründe für die Fehler kann zum einen sein, dass die Messuhren sehr ungenau sind, da diese immer wieder mit dem Zeiger verhacken und sich immer wieder umstellen, wodurch 
diese immer wieder genullt werden müssen. Daher mussten unter anderem die Gewichte immer wieder runtergenommen und wieder drauf getan werden, wodurch ebenfalls eine Fehlerquelle
entsteht.  Zudem hat schon leichtes wackeln am Tisch für große Schwankungen an den Messuhren gesorgt.
Zudem ist die Waage relativ ungenau, denn wenn die Gewichte als genormt angenommen werden, hat die bei einem genormten Gewicht von 500g ein Gewicht von 498.8g angezeigt. 

\noindent
Eine weitere große Fehlerquelle, welche den großen Fehler beim beidseitigen einhängen erklären könnte, ist dass die Stäbe niccht wirklich auf beiden Seiten eingespannt wurden,
sondern lediglich aufgelegt, wodurch ein großer systematischen Fehler entstanden sein könnte. Es lässt sich aus dem Versuch schließen, dass eine Berechnung des Elastizitätsmodul 
durch beidseitiges einspannen der Stangen nicht empfehlenswert ist.

%    \text{cr}:& E_\text{cr} =1.41 \pm 0.06    \, \si{\newton\per\meter²}
%    \text{mr}:& E_\text{mr} = 0.907 \pm 0.012 \, \si{\newton\per\meter²}
%    \text{ms}:& E_\text{ms} = 1.468 \pm 0.018 \, \si{\newton\per\meter²} 
%    \text{as}:& E_\text{as} = 0.783\pm 0.029  \, \si{\newton\per\meter²} 
%
%    \text{cr}:& E_\text{cr} = 2.75 \pm 0.23\, \si{\newton\per\meter²}
%    \text{mr}:& E_\text{mr} = 5.7  \pm 2.7 \, \si{\newton\per\meter²}
%    \text{ms}:& E_\text{ms} = 4.83 \pm 0.03\, \si{\newton\per\meter²} 
%    \text{as}:& E_\text{as} = 2.5  \pm 0.5 \, \si{\newton\per\meter²}


