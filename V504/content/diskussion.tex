\section{Diskussion}
\subsection{Literaturvergleich}
Um die ermittelte Austrittsarbeit einordnen zu können ist es sinnvoll diesen mit dem Literaturwert zu vergleichen. Diese ist gegeben als $W_\text(A) = \SI{4,54}{\eV}$ (\cite{formel-sammlung}). Um die Werte 
zu vergleichen wird die Formel 
\begin{equation*}
    \Delta p = \frac{W_1-W_2}{W_1}
\end{equation*}
\noindent
benutzt, wobei man $W_1$ und $W_2$ die zu untersuchenden Zahlen sind und $\Delta p$ die Abweichung von $W_1$ zu $W_2$ angibt.

\begin{align*}
    \Delta p &= 0.0396 \pm 0.0088 = 3.96\% \pm 0.88\% \\
\end{align*}
\noindent
Ein Grund für eine Abweichung vom Literturwert um $ 3 \%$ kann sein, dass die von uns errechnete Temperatur etwas niedriger war als sie eigentlich hätte sein sollen, wodurch sich dieser
Fehler in der Berechnung der Austrittsarbeit fortpflanzte. Da zudem der Temperaturterm in der Berechnung einmal linear und einmal quadratisch auftaucht, können kleine Abweichungen zu großeren
Fehlern führen.


\subsection{allgemeine Fehlerquellen}
Während der Durchführung des Versuchs sind einige Fehlerquellen aufgefallen, die die Ergebnisse beeinflussen.
Im Allgemeinen ensteht ein Fehler durch ablesen der Messdaten von analogen Anzeigen, 
sowie das einstellen der Saugspannung, da diese teilweise sehr stark schwankten.

\noindent
Außerdem ist das Nanoamperemeter, welches in Aufgabenteil c) verwendet wurde, sehr empfindlich und von der Umgebung leicht beeinflussbar,
weswegen die Messergebnisse leicht verfälscht wurden.

\noindent
Eine weitere Fehlerquelle ist möglicherweise die genäherte Gleichung (\ref{eqn:fermi2}), aus der zum Beispiel auch Gleichung (\ref{eqn:anlauf}) hergeleitet wurde, genauso wie die 
Näherung für den Sättigungsstrom, welcher ebenfalls eine Quelle für die Unsicherheit darstellen kann.






