\section{Diskussion}
\label{sec:Diskussion}
\subsection{Für p < 1 bar}
Im Allgemeinen ist die theoretische Verdampfungswärme gegeben durch 
\begin{table}[H]
    \centering
    \caption{Eine Tabelle der Theoriewerte\protect \cite{Chemie} für $L$.}
    \begin{tabular}{ S [table-format=3.0] S [table-format=5.0] }
        \toprule
        {$T \mathbin{\scalebox{1.5}/} \si{\celsius}$} & {$L_\text{Theorie} \mathbin{\scalebox{1.5} /} \si{\joule\per\mole}$}\\
        \midrule
        25 & 43990\\
        40 & 43350\\
        60 & 42483\\
        80 & 41585\\
        100& 40657\\
        120& 39684\\
        \bottomrule
    \end{tabular}
\label{tab:theo1bar}
\end{table}

\noindent
Diese lassen sich mitteln zu $\SI{41958.17}{\joule\per\mol}$.
Mithilfe der Formel 
\begin{equation*}
\Delta p = \bigg |\frac{f-g}{f} \bigg |,
\end{equation*}

\noindent
wobei p die Prozentuale Abweichung darstellt und f und g jeweilige Zahlenwerte dessen Abweichungen zu einander untersucht werden sollen. 
Somit besitzt das gemessene L eine Abweichung von $\SI{33,26}{\percent}$.


\noindent
Ein Grund für den Fehler kann an Messungenauigkeiten des Theromstats, sowie der schlechten Ablesbarkeit liegen, zum anderen daran, dass nicht die Werte glecihzeig abgelesen wurden, sondern
hintereinander und somit, besonders am Anfang der Messreihe als die Temperatur schnell anstieg, ebenfalls eine Fehler erzeugt wird. Weitere Gründe für die Abweichungen könnten Undichtigkeiten
im Versuchsaufbau sein.



