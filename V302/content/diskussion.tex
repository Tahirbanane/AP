\newpage
\section{Diskussion}
\label{sec:Diskussion}
Während der Durchführung des Versuchs sind einige Fehlerquellen aufgefallen, die die Ergebnisse beeinflussen.
Schon bei Aufgabe a) (\ref{subsec:aa}) fiehl auf, dass die Brückenspannung $\text{U}_\text{Br}$ nicht verschwindet,
sondern immer $\text{U}_\text{Br} > 0$ gilt. 
Daher ist das Verhältis von $\text{R}_3$ und $\text{R}_4$ nicht exakt bestimmbar.

\noindent
Auch beim ablesen der der Widerstände $\text{R}_3$ und $\text{R}_4$ vom Potentiometer gibt es eine Unsicherheit.

\noindent
Dabei hat das Potentiometer selbst auch eine Toleranz von $\pm 3\%$

\noindent
Bei Aufgabe e) (\ref{subsection:ae}) waren bei der Wien-Robinson-Brücke (\ref{fig:wien}) zwei gleiche Kapazitäten gefordert,
jedoch waren keine zwei gleichen verfügbar. 
Dies fälscht die Ergebnisse zusätzlich ab.

\noindent
Außerdem ist die Abmessung der doppelten Amplituden mit dem Cursor ungenau, da dieser manuell bedient wurde und es deswegen zu kleinen Abweichungen kommen kann.

\noindent
Da Aufgaben b)-d) nicht selbst durchgeführt werden konnten, könnnen hier keine Aussagen zu Auffälligkeiten bei der Durchführung getroffen werden.
Dennoch ist bekannt, dass die variablen Widerstände $\text{R}_2$ aus Schaltung (\ref{fig:kap}) bei Aufgabe b), sowie $\text{R}_3$ und $\text{R}_4$ aus Schaltung (\ref{fig:max}) bei Aufgabe d) eine Toleranz von $\pm 3\%$ haben.

\noindent 
Möglich ist auch ein Fehler durch Schwankungen der an der Brücke anliegenden Spannung aufgrund von technischen Mängeln der Spannungsquelle.


\noindent 
Die große Abweichung bei der Auswertung der Maxwell-Brücke von genau einer Großenodnung ist wahrscheinlich auf einen Abschreibfehler von dem Widerstand $R_2$ zurückzuführen, da es beim Versuchsaufbau
keinen $100 \si{\ohm}$, aber $1000 \si{\ohm}$ Widerstand gab.