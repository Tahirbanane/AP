\newpage
\section{Diskussion}
\label{sec:Diskussion}
\noindent
Nun werden die Ergebnisse der Untersuchung des Emissionsspektrums einer Kupfer-Röntgen-Röhre aus Kapitel (\ref{sub:emi}) diskutiert.
Im Vergleich mit den in Kapitel (\ref{sub:emilit}) aufgeführten Literaturwerten ergeben sich die in Tabelle (\ref{tab:diskemi}) befindenden relativen Abweichungen.

\begin{table}
    \centering
    \caption{Relative Abweichungen der Messwerte von den Literaturwerten.}
    \sisetup{table-format=2.1}
    \begin{tabular}{c c c c}
    \toprule
    $\Delta_{rel}\text{E}_\alpha \,/\, \%$ & $\Delta_{rel}\theta_\alpha \,/\, \% $ & $\Delta_{rel}\text{E}_\beta \,/\, \%$ &$\Delta_{rel}\theta_\beta \,/\, \% $\\
    \midrule 
    0.03& 0.099& 0.11& 0.04 \\
    \bottomrule
    \end{tabular}
    \label{tab:diskemi}
    \end{table}

\noindent
Aufgrund der geringen Abweichungen ist diese Messung als gelungen und aussagekräftig zu betrachten.

\noindent
Bei der Bestimmung der Transmission fällt auf, dass sich die Ausgleichsgerade in Abbildung (\ref{fig:trans}) immer im Bereich der Messunsicherheit befindet.
Zudem wurde eine Korrektur der gemessenen Zählraten durchgeführt, da das Geiger-Müller-Zählrohr eine Totzeit von $90 \mu s $ hat.

\noindent
Wird die bestimmte Compton-Wellenlänge mit der Literatur verglichen, 
ergibt sich eine hohe relative Abweichung von $\Delta_{rel} \lambda_c = 55 \% $.
Ein Grund dafür kann die Messunsicherheit sein, die bei sehr kleinen Werten eine große Auswirkung haben kann.

\noindent
Der Compton-Effekt kann nicht im sichtbaren Bereich des Spektrums mit Wellenlängen von ca 380 bis 780 nm auftreten, da es sich um eine andere Größenordnung handelt.
Sichtbares Licht ist im Vergleich zur Röntgenstrahlung weniger energetisch und besitzt meist nicht genug Energie um die Bindungsenergie eines Elektrons überwinden zu können.