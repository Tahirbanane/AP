\section{Zielsetzung}
Ziel dieses Versuches ist es die Compton-Wellenlänge $\lambda_C$ zu bestimmen.

\section{Theorie}
Der Compton Effekt beschreibt den Effekt, wenn $\gamma$-Strahlung an einem Elektron gestreut wird. Dies geschieht in der Regel durch kohärente und inkohärente Streuung. 
Die kohärente Streuung entspricht dem klassischen inelastischen Stoß und die inkohärente Streuung entspricht dem elastischen Stoß. Durch letzteren wird die $\gamma$-Strahlung
fequenzverschoben, da diese durch den Stoß am Elektron Energie an dieses abgibt. 

\noindent
Wenn $\lambda_1$ die einfallende und $\lambda_2$ die auslaufende Wellenlänge ist, kann die Differenz berechnet werden über
\begin{equation}
    \Delta \lambda = \frac{h}{m_ec} \left(1 - cos \theta \right) \, .
    \label{eqn:lambda}
\end{equation}

\noindent
Dabei ist $\theta$ der Winkel im den das $\gamma$-Photon gestreut wird. Der Vorfaktor $\lambda_c = \sfrac{h}{m_ec}$ wird auch die Compton-Wellenlänge genannt. 

\noindent
Zum erzeugen der Röntgenstrahlung werden Elektronen aus einer Glühkathode auf eine Anode hin beschleunigt. Dabei entsteht beim Auftreffen der Elektronen auf die Anode
Röntgenstrahlung die sich aus der Bremsstrahlung und der charakeristischen Röntgenstrahlung zusammensetzt. \\
Die Bremsstrahlung entsteht, wie es der Name suggeriert, durch die Abbremsung des Elektrons im Coulombfeld des Atoms des Anodenmaterials und ist dardurch kontinuierlich 
und die charakeristische Röntgenstrahlung entsteht durch die Ionisierung eines Atoms in der Anode, wodurch ein Elektron, unter Aussendung eines $\gamma$-Quanten, nachrückt. Diese
Energien sind diskret. Die charakeristische Röntgenstrahlung ist somit Materialabhängig. \\
\noindent
Die Compton-Wellenlänge kann durch die Transmission und Absorbtion von Röntgenstrahlung durch Aluminium bestimmt werden. Die Transmission eines Stoffes nimmt mit zunehmender Wellenlänge
ab, wodurch ebenfalls die Intensität der Compton-Wellenlänge sinkt. 
DerTransmissionskoeffizient kann durch
\begin{align}
    T=\frac{I}{I_0} \label{eq:3}
\end{align}
berechnet werden, wobei $I_0$ die Intensität der gesamten einfallenden Strahlung und I die transmittierte Strahlung, die gerde betrachtet wird, beschreibt.\\

\noindent
Die Absorbtion durch Materie kann mit dem dem Delamber´schen Gesetz beschrieben werden. Dies ist gegeben durch 
\begin{equation}
    I = I_0 \, e^{\mu d} \, ,
    \label{eqn:Delamber}
\end{equation}

\noindent
wobei $d$ die Dicke, $\mu$ der Absorptionskoeffizient und $I_0$ die einfallende Intensität ist. Der Absorptionskoeffizient $\mu$ ist gegeben durch 
$\mu = \mu_\text{Paar} + \mu_\text{Photo} + \mu_\text{Com}$, wobei $\mu_\text{Paar}$ für den Paarbildungskoeffizienten, $\mu_\text{Photo}$  für den Photoeffektkoeffizienten und
$\mu_\text{Com}$ für den Comptoneffektkoeffizienten steht.

\noindent
Wenn Strahlung auf ein Gitter fällt, dann wird diese am Gitter gebeugt. Durch Interferenz wird lediglich im Winkel $\alpha$, auch Glanzwinkel genannt, Strahlung reflektiert.
Dieser Winkel kann durch die Bragg'sche Bedingung beschrieben werden, welche gegeben ist durch
\begin{equation}
    2 d \, sin \alpha = n \, \lambda \, ,
    \label{eqn:Bragg}
\end{equation}

\noindent
wobei $\lambda$ die gebeugte Wellenlänge und $d$ die Gitterkonstanten ist. Der ganze Effekt wird Bragg'sche Reflextion genannt.
