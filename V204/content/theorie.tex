\section{Zielsetzung}
    Mit dem Versuch "`Wärmeleitung von Metallen"' soll die Wärmeleitung von Aluminium, Messing und Edelstahl untersucht werden,
    um daraus die Wärmeleitfähigkeit dieser Metalle zu bestimmen.

    \section{Theoretische Grundlage}
      Befindet sich in einem System ein Temperaturunterschied, kommt es zu einem Wärmetransort entlang des Temperaturgefälles.
      Dies geschieht entweder durch Konvektion, Wärmestrahlung oder Wärmeleitung, wobei wir uns hier auf die Wärmeleitung beschränken.

      Betrachtet wird ein Stab der Länge L, der Querschnittsfläche A, dessen Material die Dichte $\rho$ und spezifische Wärme c hat. 
      Wenn die Enden des Stabes unterschiedlicher Temperatur sind, so fließt in der Zeit dt durch die Querschnittsfläche A die Wärmemenge

      \begin{equation}
      \symup{d}Q = - \kappa A \frac{\partial T}{\partial x} \symup{d}t .
        \label{eqn:waermemenge}
      \end{equation}

      Dabei ist $\kappa$ die vom Material abhängige Wärmeleitfähigkeit. 
      Das Minuszeichen ergibt sich daraus, dass der Wärmestrom entlang des Temperaturgefälles fließt.
      Für die Wärmestromdichte $j_\text{w}$ ergibt sich

      \begin{equation}
      j_\text{w}=-\kappa \frac{\partial T}{\partial x}
        \label{eqn:waermestromdichte}
      \end{equation}       
      
      Hieraus kann die Wärmeleitungsgleichung aus der Kontinuitätsgleichung abgeleitet werden:

      \begin{equation}
      \frac{\partial T}{\partial t} = \frac{\kappa}{\rho c} \frac{\partial^2 T}{\partial x^2}.
        \label{eqn:waermeleitungsgleichung}
      \end{equation}  

      Diese gibt die räumliche- und zeitliche Entwicklung der Temperaturverteilung an.
      Die Größe $\sigma_\text{T} = \frac{\kappa}{\rho c}$, 
      die als Temperaturleitfähigkeit bezeichnet wird, gibt an, wie schnell sich der Temperaturunterschied ausgleicht.
      Die Lösung dieser Differentialgleichung hängt von der Stabgeometrie und den Anfangsbedingungen ab.

      Wird nun ein langer Stab mit der Periode T abwechseln erhitzt und abgekühlt, breitet sich eine räumliche und zeitliche Temperaturwelle im Stab aus. 
      Diese kann wie folgt beschrieben werden: 

      \begin{equation}
      T(x,t) = T_\text{max} \symup{e}^{\sqrt{\frac{\omega \rho c}{2 \kappa}}x} cos(\omega t - \sqrt{\frac{\omega \rho c}{2 \kappa}} x)
        \label{eqn:temperaturwelle}
      \end{equation}  

      Die Phasengeschwindigkeit mit der sich die Welle fortbewegt ergibt sich zu:

      \begin{equation}
      v = \frac{\omega}{k} = \frac{\omega}{\sqrt{\frac{\omega \rho c}{2 \kappa}}} = \sqrt{\frac{2 \kappa \omega}{\rho c}}
        \label{eqn:phasengeschwindigkeit}
      \end{equation}

      Aus dem Amplitudenverhältis von $A_\text{nah}$ und $A_\text{fern}$ an zwei Messstellen $x_\text{nah}$ und $x_\text{fern}$ an der Welle wird die Dämpfung ermittelt.
      Wird nun berücksichtigt, dass $\omega = \frac{2 \pi}{T'}$ (mit Periodendauer T') und und für die Phase $\Phi = \frac{2 \pi \partial t}{T'}$ ergibt sich für die Wärmeleitfähigkeit

      \begin{equation}
      \kappa = \frac{\rho c (\Delta x)^2}{2 \Delta t \, ln (\frac{A_\text{nah}}{A_\text{fern}})}
        \label{eqn:waermeleitfaehigkeit}
      \end{equation}

      mit dem Abstand der beiden Messstellen $\Delta x$ und der Phasendifferenz $\Delta t$ der Temperaturwelle zwischen den beiden Messstellen.