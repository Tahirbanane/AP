\section{Zielsetzung}
    Im Versuch 'Wärmeleitung von Metallen' wird die Wärmeleitung von Aluminium, Messing und Edelstahl untersucht,
    um daraus die Wärmeleitfähigkeit dieser Metalle zu bestimmen.

    \section{Theoretische Grundlage}
      Bei einem System Temperaturunterschied im System kommt es zu einem Wärmetransport entlang des Temperaturgefälles.
      Hier wird unterschieden zwischen Konvektion, Wärmestrahlung und Wärmeleitung, wobei sich hier auf die Wärmeleitung beschränkt wird.

      \noindent Wenn die Enden eines Stabes der Länge L, der Querschnittsfläche A, der aus einem Material mit der spezifischen Wärme c und der Dichte $\rho$ besteht, unterschiedlicher Temperatur sind,
      fließt in der Zeit dt durch A die Wärmemenge 

      \begin{equation}
      \symup{d}Q = - \kappa A \frac{\partial T}{\partial x} \symup{d}t .
        \label{eqn:waermemenge}
      \end{equation}

      \noindent Dabei ist $\kappa$ die vom Material abhängige Wärmeleitfähigkeit. 
      Das Minuszeichen ergibt sich daraus, dass der Wärmestrom entlang des Temperaturgefälles fließt.
      Für die Wärmestromdichte $j_\text{w}$ ergibt sich

      \begin{equation*}
      j_\text{w}=-\kappa \frac{\partial T}{\partial x}
        \label{eqn:waermestromdichte}
      \end{equation*}       
      
      \noindent Hieraus kann die Wärmeleitungsgleichung aus der Kontinuitätsgleichung abgeleitet werden:

      \begin{equation*}
      \frac{\partial T}{\partial t} = \frac{\kappa}{\rho c} \frac{\partial^2 T}{\partial x^2}.
        \label{eqn:waermeleitungsgleichung}
      \end{equation*}  

      \noindent Diese gibt die räumliche- und zeitliche Entwicklung der Temperaturverteilung an.
      Dabei ist $\sigma_\text{T} = \frac{\kappa}{\rho c}$ die Temperaturleitfähigkeit, die ein Maß für die Geschwindigkeit des Temperaturausgleichs ist.

      \noindent Wenn ein Stab periodisch abwechselnd erhitzt und abgekühlt wird, breitet sich eine räumliche und zeitliche Temperaturwelle im Stab aus. 
      Diese kann wie folgt beschrieben werden: 

      \begin{equation*}
      T(x,t) = T_\text{max} \symup{e}^{\sqrt{\frac{\omega \rho c}{2 \kappa}}x} cos(\omega t - \sqrt{\frac{\omega \rho c}{2 \kappa}} x)
        \label{eqn:temperaturwelle}
      \end{equation*}  

      \noindent Die Phasengeschwindigkeit mit der sich die Welle fortbewegt ergibt sich zu:

      \begin{equation}
      \upsilon = \frac{\omega}{k} = \frac{\omega}{\sqrt{\frac{\omega \rho c}{2 \kappa}}} = \sqrt{\frac{2 \kappa \omega}{\rho c}}
        \label{eqn:phasengeschwindigkeit}
      \end{equation}

      \noindent Aus dem Amplitudenverhältis von $A_\text{nah}$ und $A_\text{fern}$ an zwei Messstellen $x_\text{nah}$ und $x_\text{fern}$ an der Welle wird die Dämpfung ermittelt.
      Wird nun berücksichtigt, dass $\omega = \frac{2 \pi}{T'}$ (mit Periodendauer T') und für die Phase $\Phi = \frac{2 \pi \Delta t}{T'}$ ergibt sich für die Wärmeleitfähigkeit

      \begin{equation}
      \kappa = \frac{\rho c (\Delta x)^2}{2 \Delta t \, ln (\frac{A_\text{nah}}{A_\text{fern}})}
        \label{eqn:waermeleitfaehigkeit}
      \end{equation}

     \noindent mit dem Abstand der beiden Messstellen $\Delta x$ und der Phasendifferenz $\Delta t$ der Temperaturwelle zwischen den beiden Messstellen.