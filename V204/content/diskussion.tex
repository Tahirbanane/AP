\newpage
\section{Diskussion}
Während der Durchführung des Versuchs sind einige Fehlerquellen aufgefallen, die die Ergebnisse beeinflussen.
Beim auflegen der Isolierung auf die Probenstäbe fiel auf, dass die Isolierung die Probenstäbe nicht vollständig bedeckt.
Dadurch ist der Wärmeaustausch, der mit der Umgebung statt findet, noch schlechter zu verhindern.

\noindent Außerdem birgt das manuelle stoppen der Zeit und gleichzeitigem umlegen des Schalters von 'COOL' auf 'HEAT' bei der dynamischen Methode eine Messunsicherheit, 
da es hier auch auf die Reaktionszeit ankommt.
Daher ist davon auszugehen, dass die exakten Perioden nicht eingehalten wurden und minimal unregelmäßig gemessen wurde.

\noindent Zudem wurden die Platinen im Laufe des Versuchs zweimal ausgetauscht, da diese die Probenstäbe nicht ausreichend gekühlt haben.
Da die unterschiedlichen Platinen möglicherweise voneinander abweichende Ergebnisse hervorrufen, 
zum Beispiel durch ein unterschiedlich schnell oder ungleichmäßig heizendes Peltierelement, muss auch hier ein Fehler in betracht gezogen werden.

\noindent Möglich ist auch ein Fehler durch Schwankungen der an der Platine anliegenden Spannung aufgrund von technischen Mängeln der Spannungsquelle.

\noindent Die Ergebnisse aus (\ref{eqn:abstand})
wurden mit einem Lineal gemessen und die Messunsicherheit wurde auf $\Delta s=\pm 0.05 \si{\centi\meter}$ geschätzt.
Beim weiterrechnen mit diesen Werten pflanzt sich der Fehler natürlich fort.

\noindent
Des Weiteren gab es auch bei der Messung große Probleme.
Es wurden am Ende Messergebnisse einer anderen Gruppe verwendet, da die ermittelten Daten nicht vollständig gespeichert wurden und so alle Messungen hinfällig waren. 

\label{sec:Diskussion}

Es gab bei der Berechnung der Wärmeleitfähigkeit einen systematischen Fehler, welcher sich durch alle Rechnungen gezogen hat. Dardurch konnten keine korrekten Zahlenwerte errechnet werden.

Zudem gab es bereits bei der Messung große Probleme. Es wurden am Ende Messergebnisse einer anderen Gruppe verwendet, da die ermittelten Daten nicht vollständig gespeichert wurden und so 
alle Messungen hinfällig waren.