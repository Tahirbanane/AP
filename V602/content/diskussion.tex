\newpage
\section{Diskussion}
Da wir den Versuch nicht selbst durchgeführt, sondern lediglich die Anleitung und die Messdaten bekommen haben, können wir mögliche Fehlerquellen während der Durchführung schwer einschätzen. 
Jedoch ist eine mögliche Fehlerquelle, dass bei der Messung die Tür des Gehäuses nicht richtig geschlossen war, wodurch Fehler entstehen können.

\noindent
Eine weitere Fehlerquelle kann in der Abschätzung zur Berechnung der Abschirmkonstante $\sigma_\text{L}$ liegen, da für die Berechnung dieser die $L_I$ - Kante, aufgrund der fehlenden 
Auflösung, nicht in die Berechnung mit eingeflossen ist.

\noindent
Bei der Untersuchung der Bragg Bedingung aus Kapitel (\ref{sub:bragg}) beträgt die absolute Abweichung vom Sollwinkel 0.2° und liegt somit deutlich unter dem Toleranzwert von 1°.

\noindent
Nun werden die Ergebnisse der Untersuchung des Emissionsspektrums einer Kupfer-Röntgen-Röhre aus Kapitel (\ref{sub:emi}) diskutiert.
Im Vergleich mit den in Kapitel (\ref{sub:emilit}) aufgeführten Literaturwerten ergeben sich die in Tabelle (\ref{tab:diskemi}) befindenden relativen Abweichungen.

\begin{table}
    \centering
    \caption{Relative Abweichungen der Messwerte von den Literaturwerten.}
    \sisetup{table-format=2.1}
    \begin{tabular}{c c c c}
    \toprule
    $\Delta_{rel}\text{E}_\alpha \,/\, \%$ & $\Delta_{rel}\theta_\alpha \,/\, \% $ & $\Delta_{rel}\text{E}_\beta \,/\, \%$ &$\Delta_{rel}\theta_\beta \,/\, \% $\\
    \midrule 
    0.03& 0.099& 0.11& 0.04 \\
    \bottomrule
    \end{tabular}
    \label{tab:diskemi}
    \end{table}

\noindent
Aufgrund der geringen Abweichungen ist diese Messung als gelungen und aussagekräftig zu betrachten.
Aus den gegebenen Messwerten kann die minimale Wellenlänge $\lambda_{min}$ nicht bestimmt werden,
da der Grenzwinkel nicht ermittelt werden kann.
Der theoretisch bestimmte Grenzwinkel von $\theta_{Grenz} = 5.045°$ erscheint realistisch, da der Kurvenverlauf um diesen Winkel zwar außerhalb des Intervalls von Abbildung (\ref{fig:emi}) liegt,
aber ein Abfall auf Null dort möglich ist.

\noindent
Im Folgenden werden die Messwerte aus Tabelle (\ref{tab:mess3}) mit den Literaturwerten aus Tabelle (\ref{tab:Glanz}) verglichen.
Dafür werden die relativen Abweichungen in Tabelle (\ref{tab:disk}) notiert.

\begin{table}
    \centering
    \caption{Relative Abweichungen der Messwerte von den Literaturwerten.}
    \sisetup{table-format=2.1}
    \begin{tabular}{c c c c}
    \toprule
         &  $\Delta_{rel}\text{E}_\text{K} \,/\, \%$ & $\Delta_{rel}\theta_\text{K} \,/\, \% $ & $\Delta_{rel}\sigma_\text{K} \,/\, \% $\\
    \midrule 
      Zn & 0.003& 0.002& 0.15 \\
      Ga & 0.19 & 1.35 & 1.2  \\
      Br & 0.07 & 0.69 & 0.35 \\
      Rb & 0.14 & 0.14 & 0.45 \\
      Sr & 1.57 & 1.62 & 6.57 \\
      Zr & 1.47 & 1.51 & 6.4  \\
    \bottomrule
    \end{tabular}
    \label{tab:disk}
    \end{table}

\noindent
Der Tabelle ist zu entnehmen, dass allgemein nur kleine Abweichungen auftreten.
Wird $\Delta_{rel}\theta_\text{K}$ betrachtet, fällt auf, 
dass die Abweichungen für Zinn, Brom und Robidium im Rahmen der in Kapitel (\ref{sub:bragg}) ermittelten Abweichung liegen.
Gallium, Strontium und Zirkonium liegen knapp über der ermittelten Abweichung von 0.7\%.
Auffällig ist, dass die Abweichung bei größer werdender Ordnungszahl zunimmt und bei Strontium und Zirkonium deutlich höhere Fehler,
vorallem bei der Abschirmkonstante, auftreten.


\noindent
Wird die Rydbergkonstante mit der Literatur verglichen, fällt auf, dass die absolute Abweichung $\Delta_{\text{abs}} R_y= 1142292.94 \frac{1}{\text{m}}$ und die relative Abweichung $\Delta_{\text{rel}} R_y= 10.41\%$
betragen.
Diese hohe Abweichung könnte der quantitativen Ausgleichsgeraden geschuldet sein.