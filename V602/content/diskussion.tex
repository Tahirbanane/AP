\section{Diskussion}
Da wir den Versuch nicht selbst durchgeführt, sondern lediglich die Anleitung und die Messdaten bekommen haben, können wir mögliche Fehlerquellen während der Durchführung schwer einschätzen. 
Jedoch ist eine mögliche Fehlerquelle, dass bei der Messung die Tür des Gehäuses nicht richtig geschlossen war, wodurch Fehler entstehen können.

\noindent
Eine weitere Fehlerquelle kann in der Abschätzung zur Berechnung der Abschirmkonstante $\sigma_\text{L}$ liegen, da für die Berechnung dieser die $L_I$ - Kante, aufgrund der fehlenden 
Auflösung, nicht in die Berechnung mit eingeflossen ist.

\noindent
Wird die Rydbergkonstante mit der Literatur verglichen, fällt auf, dass die absolute Abweichung $\Delta_{\text{abs}} R_y= 869904.86 \frac{1}{\text{m}}$ und die relative Abweichung $\Delta_{\text{rel}} R_y= 7.93\%$
betragen.
Diese Abweichung könnte der quantitativen Ausgleichsgeraden geschuldet sein.