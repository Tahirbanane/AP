\section{Diskussion}
\label{sec:Diskussion}
Während der Durchführung des Versuchs sind einige Fehlerquellen aufgefallen, die die Ergebnisse beeinflussen.

\noindent
Sowohl die Amplituden und Zeiten in Aufgabenteil a) als auch die Kondensator- und Erregerspannung in Aufgabenteil c) 
wurden direkt am Oszillographen vermessen.
Dadurch ensteht eine Ungenauigkeit bei diesen Größen.

\noindent
Der in Aufgabenteil b) gesuchte Widerstand $R_{ap}$ wird durch eine optische Abschätzung der Kurve ermittelt,
sodass nur ein ungefähres Ergebnis erzielt werden kann.

\noindent
Außerdem besitzt jeder Aufbau einen Gesamtwiderstand, der größer ist als die Summe aller angegebenen Widerstände.
Dies kommt daher, dass auch die Kabel einen Widerstand haben und die Geräte selbst noch einen Innenwiderstand besitzen.

\noindent
Zudem fehlten bei der Auswertung der Güte Messdaten, sodass diese schon bei kleinen Messfehlern extrem stark schwankte. 