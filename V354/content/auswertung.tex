\newpage
\section{Auswertung}

In diesem Abschnitt wird der Versuch ausgewertet.

\subsection{gedämpften Schwingung}
Zunächst wird bei einer gedämpften Schwingung die Zeitabhängigkeit der Amplitude untersucht, um den effektiven Dämpfungswiderstand bestimmen zu können.
Dabei wird zunächst die in der Messung bestimmte Amplitudenspitzen in die Tabelle (\ref{tab:1}) eingetragen.

\begin{table}
    \centering
    \caption{Gemessene Spannungsamplituden in Abhängigkeit von der Zeit}
    \label{tab:1}
    \begin{tabular} {S[table-format=4.1] S[table-format=2.2]}
        \toprule
        {$t \mathbin{/} \si{\micro\second}$} & {$U \mathbin{/} \si{\volt}$}  \\
    \midrule
    0 	    & \,\,  -82   \\
    12.5 	& \,\,   76   \\
    25.5	& \,\,  -64   \\
    40  	& \,\,   56   \\
    55	    & \,\,  -48   \\
    67.5	& \,\,   40   \\	
    80	    & \,\,  -36   \\
    93.75	& \,\,   32   \\
    107.5	& \,\,  -26   \\
    122.5	& \,\,   22   \\
    135	    & \,\,  -20   \\
    148.75  & \,\,	 16   \\
    162.5	& \,\,  -14   \\
    175	    & \,\,   11   \\
    188.75  & \,\,	-10   \\
    202.5	& \,\,   9    \\
    216.25  & \,\,	-8    \\
    \bottomrule
\end{tabular}
\end{table}

\noindent
Es reicht jedoch die Einhüllende zu betrachten. Da die Einhüllende unterhalb der X-Achse und Oberhalb sich um einen Vorzeichen unterscheiden, kann 
zur Modelierung dieser, lediglich der Betrag der gemessen Spannung betrachtet werden. Daraus folgen die Abbildungen (8), in welcher die Zeit gegen die Spannung und gegen den 
Spannungsbetrag aufgetragen werden.

\newpage
\begin{figure}
    \centering
    \includegraphics{Daten/a.pdf}
    \label{fig:1}
    \caption{Gemessene Spannungsamplituden mit Ausgleichsfunktion}
\end{figure}

Um die Einhüllende zu modelieren, kann diese in der Form
\begin{equation}
    U = a \symup{e}^{b*t}
\end{equation}

\noindent
angenommen werden. Aus der Theorie gilt mit der \autoref{eqn:einh} der Zusammenhang $a = U_0$ und $b = -2 \pi \mu$. Daraus folgt %U_0 e^-2*pi*my*t


\begin{align*}
    a &= 0.0846 \pm 0.0008 \Leftrightarrow U_0 = 0.0846 \pm 0.0008 \si{\volt} \\
    b &= -10852.9624 \pm 174.9926 \Leftrightarrow \mu = 1727.3026 \pm 27.8509 \si{\per\second} \, .\\
\end{align*}

\noindent
Mit der Beziehung 
\begin{equation}
    R_\text{eff} = 4\pi\mu L = -2 L b
\end{equation}

\noindent
lässt sich der Dämpfungswiderstand R, mit $L = 3.5 \pm 0.01 \si{\milli\henry}$ zu
\begin{equation*}
    R_\text{eff} = 76 \pm 1,2 \si{\ohm}
\end{equation*}
bestimmen. Damit ist die Abweichung des verbauten Widerstands $R = 30,3 \pm 0,1 \si{\ohm}$ $99.749 \pm 00.004 \si{\percent}$.

\noindent
Die Abklingdauer kann mithilfe der Formel 
\begin{equation*}
    T_\text{ex} = \frac{1}{2\pi\mu} = - \frac{1}{b}
\end{equation*}
berechnet werden zu
\begin{align*}
    T_\text{ex} = 92,1 \pm 1,5 \si{\micro\second} \, . \\
\end{align*}



\subsection{aperiodische Grenzfall}
Zusätzlich lässt sich der Dämpfungswiderstand $R_\text{ap}$ experimentell bestimmen zu $R_ap = 12,725 \si{\ohm}$. Der theoretische Wert lässt sich mit der Formel (\ref{eqn:rap}) %R = 2(L/C)^1/2
mit den gegeben Werten
\begin{align*}
   R &= 30,3 \pm 0,1 \, \si{\ohm} \\
   L &= 3,5 \pm 0,01 \, \si{\milli\henry} \\
   C &= 5 \pm 0 \, \si{\nano\farad} \\
\end{align*}

\noindent
aus dem Aufbau bestimmen. Daraus folgt das $$ R_\text{ap} = 1673.3 \pm 2.4 \, \si{\ohm} $$ und eine Abweichung von $ 660.46 \, \si{\percent}$.

\subsection{Frequenzabhängigkeit der Kondensatorspannung}

Die gemessene Kondensatorspannung $U_C$ und die dazugehörige Frequenz $\omega$ wird in die \autoref{tab:2} eingetragen und 
der Quotient $\frac{U_C}{U_\text{Err}}$ gegen die Frequenz $\omega$ halblogarithmisch in die Abbildung (9) aufgetragen.

\begin{table}
    \centering
    \caption{Messwerte zur Kondensator- und Erregerspannung}
    \label{tab:2}
    \begin{tabular} {S[table-format=2.2] S[table-format=1.2] S[table-format=1.2] S[table-format=1.2]}
        \toprule
        {$\nu \mathbin{/} \si{\kilo\hertz}$}&
        {$U_\text{C} \mathbin{/} \si{\volt}$} & {$U_\text{Err} \mathbin{/} \si{\volt}$}   \\
        \midrule
        10		    &        50			&         50 \\
        15		    &        55			&         50 \\
        20		    &        65			&         50 \\
        25		    &        87.5		&         50 \\
        30		    &        130		&	      50 \\
        32		    &        177.5		&         50 \\
        34		    &        265		&	      40 \\
        36		    &        430		&	      30 \\
        38		    &        405		&	      30 \\
        40		    &        235		&	      40 \\
        42		    &        150		&	      50 \\
        45		    &        97.5		&         50 \\
        50		    &        55			&         50 \\
        55		    &        40			&         50 \\
        58		    &        32			&         50 \\		
        \bottomrule
    \end{tabular}
\end{table}

\begin{figure}
    \centering
    \includegraphics{Daten/c.pdf}
    \label{fig:21}
    \caption{Quotient $\frac{U_C}{U_\text{Err}}$ gegen die Frequenz $\omega$ halblogarithmisch aufgetragen}
\end{figure}


\noindent
Aus der Grafik lässt sich für die Frequenzen, bei denen die Spannung den Wert $\frac{U_\text{max}}{U_\text{Err}}\frac{1}{\sqrt{2}}$ erreicht, entnehmen als

\begin{align*}
    \omega_+ &= 35,1 \pm 0,3 \,  \si{\kilo\hertz} \\
    \omega_- &= 38,5 \pm 0,3 \,  \si{\kilo\hertz} \, .
\end{align*}

\noindent
Daraus folgt die Breite 
\begin{equation*}
    b = 3,4 \pm 0,3 \si{\kilo\hertz} \, .
\end{equation*}

\noindent
Für die Güte kann mit der \autoref{eqn:guete} $$ q = 70 \pm 6 \, $$ berechnet werden. 

\noindent
Die theoretische Breite der Ressonanzkurve lässt sich der \autoref{eqn:wmp} nähern. Mithilfe dieser Näherung lässt sich die Breite der Ressonanzkurve und die Güte berechnen zu %w+ - w- = R/L

\begin{align*}
    b & = 8.657 \pm 0.025 \, \si{\kilo\hertz}\\
    q & = 27.61\pm 0.04 \, . 
\end{align*}

\noindent
Daraus folgt für den Fehler zwischen der theoretischen und gemessen Güte ein Fehler von $1.55 \pm 0.22 \si{\percent}$.

