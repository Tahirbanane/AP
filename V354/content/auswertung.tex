\newpage
\section{Auswertung}

In diesem Abschnitt wird der Versuc hausgewertet.

\subsection{gedämpften Schwingung}
Zunächst wird bei einer gedämpften Schwingung die Zeitabhängigkeit der Amplitude untersucht, um den effektiven Dämpfungswidesdtand bestimmen zu können.
Dabei wird zunächst die in der Messung bestimmte Amplitudenspitzen in die \autoref{tab:1} eingetragen.

\begin{table}
    \centering
    \caption{Gemessene Spannungsamplituden in Abhängigkeit von der Zeit}
    \label{tab:1}
    \begin{tabular} {S[table-format=4.1] S[table-format=2.2]}
        \toprule
        {$t \mathbin{/} \si{\micro\second}$} & {$U \mathbin{/} \si{\volt}$}  \\
    \midrule
    0 	    & \,\,  -82   \\
    12.5 	& \,\,   76   \\
    25.5	& \,\,  -64   \\
    40  	& \,\,   56   \\
    55	    & \,\,  -48   \\
    67.5	& \,\,   40   \\	
    80	    & \,\,  -36   \\
    93.75	& \,\,   32   \\
    107.5	& \,\,  -26   \\
    122.5	& \,\,   22   \\
    135	    & \,\,  -20   \\
    148.75  & \,\,	 16   \\
    162.5	& \,\,  -14   \\
    175	    & \,\,   11   \\
    188.75  & \,\,	-10   \\
    202.5	& \,\,   9    \\
    216.25  & \,\,	-8    \\
    \bottomrule
\end{tabular}
\end{table}

\noindent
Es reicht jedoch lediglich sich die einhüllende zu betrachten. Da jedoch die einhüllende unterhalb der X-Achse und Oberhalb sich lediglich um einen Vorzeichen unterschreiben, kann 
zur Modelierung dieser, lediglich der Betrag der gemessen Spannung betrachtet werden. Daraus folgen die Grafiken (\ref{fig:1}), in welcher der die Zeit gegen die Spannung und gegen den 
Spannungsbetrag aufgetragen.

\newpage
\begin{figure}
    \centering
    \label{fig:1}
    \includegraphics{Daten/a.pdf}
    \caption{Gemessene Spannungsamplituden mit Ausgleichsfunktion}
\end{figure}

Um die einhüllende zu modelieren, kann diese in der Form
\begin{equation}
    U = a \symup{e}^{b*t}
\end{equation}

\noindent
angenommen werden.Aus der Theorie gilt mit der \autoref{} der Zusammenhang $a = \pm U_0$ und $b = -2 \pi \mu$. Daraus folgt %U_0 e^-2*pi*my*t


\begin{align*}
    a &= 0.0846 \pm 0.0008 \Leftrightarrow U_0 = 0.0846 \pm 0.0008 \si{\volt} \\
    b &= -10852.9624 \pm 174.9926 \Leftrightarrow \mu = 1727.3026 \pm 27.8509 \si{\per\second} \, .\\
\end{align*}

\noindent
Mit der Beziehung 
\begin{equation}
    R_\text{eff} = 4\pi\mu L = -2 L b
\end{equation}

\noindent
lässt sich der Dämpfungswiderstand R, mit $L = 3.5 \pm 0.01 \si{\milli\henry}$ zu
\begin{equation*}
    R_\text{eff} = 76 \pm 1,2 \si{\ohm}
\end{equation*}

\noindent
bestimmen. Damit ist die Abweichung vom im verbauten Widerstand $R = 30,3 \pm 0,1 \si{\ohm}$ $151 \pm 0.04 \si{\percent}$.

\noindent
Die Abklingdauer kann mithilfe der Formel 
\begin{equation*}
    T_\text{ex} = \frac{1}{2\pi\mu} = - \frac{1}{b}
\end{equation*}
berechnet werden zu
\begin{align*}
    T_\text{ex} = 92,1 \pm 1,5 \si{\micro\second} \, . \\
\end{align*}



\subsection{aperiodische Grenzfall}
Zusätzlich lässt sich der Dämpfungswiderstand $R_\text{ap}$ experimentell bestimmen zu $R_ap = 12,725 \si{\kilo\ohm}$. Der theoretische Wert lässt sich mit der Formel (\ref{}) %R = 2(L/C)^1/2
mit den gegeben Werten
\begin{align*}
   R &= 30,3 \pm 0,1 \, \si{\ohm} \\
   L &= 3,5 \pm 0,01 \, \si{\milli\henry} \\
   C &= 5 \pm 0 \, \si{\nano\farad} \\
\end{align*}

\noindent
aus dem Aufbau. Daraus folgt das $$ R_\text{ap} = 1673.3 \pm 2.4 \, \si{\ohm} $$ und eine Abweichung von $ 660.46 \, \si{\percent}$.

\subsection{Frequenzabhängigkeit der Kondensatorspannung}

Die gemessene Kondensatorspannung $U_C$ und die dazugehörige Frequenz $\omega$ wird in die \autoref{tab:2} eingetragen und 
der Quotient $\frac{U_C}{U_\text{Err}}$ gegen die Frequenz $\omega$ halblogarithmisch in die \autoref{fig:2} aufgetragen.



