\section{Durchführung}
Das Experiment wird nach Abbildung (\ref{fig:aufbau}) mit einer $\ce{^{204}Tl}$ Quelle aufgebaut.




\subsection{Aufgabe a)}
Die $\beta$-Quelle wird vor dem Eintrittsfenster des Zählrohres platziert.
Dabei wird die Zählrate von $100\frac{\text{Imp}}{\text{s}}$ bei mittlerer Spannung nicht überschritten, um eine Totzeit-Korrektur zu vermeiden.
Nun wird die Zählrate für Spannungen zwischen $300\si{\volt}$ und $700\si{\volt}$ im Abstand von $\Delta \text{U} =10 \si{\volt}$ für jeweils $60\si{\second}$ gemessen und notiert.
Die Werte wurden in Tabelle (\ref{tab:a}) zusammengetragen.


\subsection{Aufgabe c)}
Die Totzeit wird nun auf 2 verschiedene Arten bestimmt.
\begin{enumerate}
\item \textbf{Oszilloskop}

Die Quelle wird an das Zählrohr gestellt, damit eine hohe Strahlenintensität in das Zählrohr eintritt.
Nun kann auf dem Oszilloskop eine Kurve nach Abbildung (\ref{fig:totzeit}) beobachtet werden, aus der die Totzeit abgelesen wird.

\item \textbf{Zwei-Quellen-Methode}

Die genauere Methode ist die Zwei-Quellen-Methode.
Hierzu wird die $\ce{^{204}Tl}$ Quelle näher an das Zählrohr gestellt, um eine Totzeit-Korrektur zu erhalten ($\text{N}_1$).
Danach wird eine zweite Quelle im gleichen Abstand auf das Zählrohr gerichtet ($\text{N}_{1+2}$).
Anschließend wird die $\ce{^{204}Tl}$ Quelle entfernt ($\text{N}_2$).
Die Messzeit beträgt bei allen drei Messungen $\text{t}=120 \si{\second}$, um die Genauigkeit zu erhöhen.
Dabei wurden folgende Zählraten gemessen:

\begin{align*}
\text{N}_1 &= 96041 \, \frac{\text{Imp}}{120\si{\second}} \\
\text{N}_{1+2} &= 158479 \, \frac{\text{Imp}}{120\si{\second}} \\
\text{N}_2 &= 76518 \, \frac{\text{Imp}}{120\si{\second}}
\label{n:n}
\end{align*}

\end{enumerate}

\subsection{Aufgabe d)}
Der Zählrohrstrom kann parallel zur Aufnahme der Messwerte in a) notiert werden.
Alle $50 \si{\volt}$ wird am Mikro-Amperemeter der Zählrohrstrom abgelesen.
Die Werte sind in Tabelle (\ref{tab:d}) eingetragen.
Es gilt $\Delta \text{t}=60\si{\second}$.