\section{Diskussion}
Bei diesem Versuch sind Fehlerquellen und Unsicherheiten, die die Ergebnisse verfälschen, unumgänglich.

\noindent
Die Integrationszeit, über die in allen drei Aufgaben gemessen wurde, wurde manuell gestartet und gestoppt.
Dadurch kommt es zu einer Unsicherheit aufgrund der individuellen Reaktionszeit.

\noindent
Die dabei gemessenen Zählraten sind Poisson verteilt, d.h. es gibt eine Messunsicherheit von $\Delta\text{N}=\sqrt{\text{N}}$.

\noindent
Bei der oszillographischen Totzeit Bestimmung gilt das Ergebnis als verfälscht, da die Totzeit vom Bildschirm des Oszilloskops abgelesen wird.

\noindent
Ber der Zwei-Quellen-Methode entsteht eine Messunsicherheit, da nicht sichergestellt werden kann, dass sich die beiden Präparate in genau dem gleichen Abstand relativ zum Eintrittsfensters des Zählrohrs befinden.
Außerdem ist die Formel (\ref{eqn:t}) zur Berechnung der Totzeit genähert.

\noindent
Zur Bestimmung der freigesetzten Ladungsmenge wurde der Zählrohrstrom von einem Amperemeter abgelesen.
Die Ablesegenauigkeit beträgt $\Delta\text{I}=0,05\si{\micro\ampere}$

\noindent
Diese Faktoren führen dazu, dass theoretische Werte von praktisch gemessenen Werten abweichen und die Ergebnisse verfälscht wurde.


