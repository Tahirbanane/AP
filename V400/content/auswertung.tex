\newpage
\section{Auswertung}
Im diesem Kapitel wird der Versuch ausgewertet. Bei jeder durchgeführten Messung besteht eine Messunsicherheit von $\Delta \alpha \, = 0.5 °$, zusätzlich wurden alle 
berechneten Unsicherheiten mit Python mit dem Paket uncertainties\cite{uncertainties}.

\subsection{Reflextion}
Um das Reflextionsgesetz $\alpha_1 \, = \alpha_2$ zu überprüfen werden für 7 Einfallswinkel $\alpha_1$ die Reflextionswinkel $\alpha_2$ bestimmt.
Die Messergebnisse befinden sich in der \autoref{tab:ref}. 

\begin{table}
    \centering
    \caption{Messwerte des Einfallswinkels $\alpha_1$ und dem Reflextionswinkel $\alpha_2$}
    \label{tab:ref}
    \sisetup{table-format=2.1}
    \begin{tabular}{c c}
    \toprule
         $\alpha_1 \, / \, ° $ & $\alpha_2 \, / \, ° $\\
    \midrule 
    20 & 20\\
    30 & 30\\
    40 & 40\\
    45 & 45\\
    50 & 50\\
    60 & 60\\
    70 & 70\\
    \bottomrule
    \end{tabular}
\end{table}

\noindent
Die Messergebnisse bestätigen ziemlich gut das Reflextionsgesetz.

\subsection{Brechungsgesetz}
Das Brechungsgesetz wird an der planparallelen Platte untersucht und der Brechungsindex wird an der Winkelskala abgelesen. Die Messergebnisse befinden sich in der \autoref{tab:bre}. 
\begin{table}
    \centering
    \caption{Messwerte des Einfallswinkels $\alpha$, dem Brechungswinkel $\beta$ und dem Brechungsindex $n$}
    \label{tab:bre}
    \sisetup{table-format=2.1}
    \begin{tabular}{c c c}
    \toprule
         $\alpha \, / \, ° $ & $\beta \, / \, °$ & Brechungsindex $n$\\
    \midrule
    10 & 7    & 1.42 $\pm$ 0.10\\
    20 & 13,5 & 1.47 $\pm$ 0.05\\
    30 & 20   & 1.462$\pm$ 0.035\\
    40 & 25   & 1.521$\pm$ 0.028\\
    50 & 31   & 1.487$\pm$ 0.022\\
    60 & 35,5 & 1.491$\pm$ 0.018\\
    70 & 39   & 1.491$\pm$ 0.018\\
    \bottomrule
    \end{tabular}
\end{table}

\noindent
Aus der \autoref{} und dem Brechungsindex $n_1 = 1$ und $n_2 = n$ kann der Zusammenhang 
\begin{equation}
    \frac{sin \, \alpha}{sin \, \beta} = n
\end{equation}

\noindent
daraus lässt sich dann der Brechungsindex für die 7 Messergebnisse berechnen und mitteln. Die 7 berechneten Brechungsindezes befinden sich ebenfalls in der \autoref{tab:bre}.
Alle Werte gemittelt ergeben sich zu 
\begin{align*}
    n = 1,478 \pm 0,018
\end{align*} 

Zusätzlich lässt sich mit Hilfe der \autoref{eqn:snel} die Geschwindigkeit im Plexiglas bestimmen. Dazu wird die Gleichung nach $v_2$ umgestellt, somit ergibt sich
\begin{equation}
    v_2 = \frac{v_1}{n} \, .
\end{equation}

\noindent
Wenn $n = 1.478 \pm 0.018$ und $v_1 = c = 2,9979 \cdot 10^8 \si{\meter\per\second}$ ergibt sich für die Ausbreitungsgescchwindigkeit
\begin{align*}
    v_2 = 2,029 \cdot 10^8 \, \pm \, 0,025 \cdot 10^8 \, \si{\meter\per\second} \, .
\end{align*}

\subsection{Strahlversatz}
Der Strahlversatz $s$ durch einen Körper mit planparallelen Flächen ist gegeben durch die Formel
\begin{equation}
    s = d \, \frac{sin(\alpha - \beta)}{cos \, \beta} \, .
\end{equation}

\noindent
Mithilfe dieser Gleichung lässt sich der Strahlversatz für jedes einzelne Messpaar berechnen. Der Strahlversatz für jedes einzelne Messpaar 
befindet sich in der \autoref{tab:str}.

\begin{table}
    \centering
    \caption{Messwerte des Einfallswinkels $\alpha$, dem Brechungswinkel $\beta$ und dem berechneten Strahlenversatz $s$}
    \label{tab:str}
    \sisetup{table-format=2.1}
    \begin{tabular}{c c c}
    \toprule
         $\alpha \, / \, ° $ & $\beta \, / \, °$ & Strahlenversatz $s  \, / \, \si{\centi\meter}$\\
    \midrule
    10 & 7    & 0,31  $\pm$  0,05\\
    20 & 13,5 & 0,68  $\pm$  0,05\\
    30 & 20   & 1,08  $\pm$  0,05\\
    40 & 25   & 1,67  $\pm$  0,05\\
    50 & 31   & 2,22  $\pm$  0,04\\
    60 & 35,5 & 2,98  $\pm$  0,04\\
    70 & 39   & 3,877 $\pm$  0,029\\
    \bottomrule
    \end{tabular}
\end{table}

\noindent
Zum Vergleich kann der Winkel $\beta$ auch mit dem im vorherigen Kapitel berechneten Brechungsindex über das Gesetz von Snellius berechnet werden und zwar abgelesen
\begin{equation}
    \beta = arcsin\left(\frac{sin(\alpha)}{n}\right) \, .
\end{equation}

\noindent
So ergibt sich für jedes $\alpha$ ein neues $\beta$ mit welchem der Strahlenversatz $s$ neu berechnet werden kann. Die errechneten Werte befinden sich in der \autoref{tab:beta}.

\begin{table}
    \centering
    \caption{Messwerte des Einfallswinkels $\alpha$, dem errechneten Brechungswinkel $\beta$ und dem berechneten Strahlenversatz $s$}
    \label{tab:beta}
    \sisetup{table-format=2.1}
    \begin{tabular}{c c c}
    \toprule
         $\alpha \, / \, ° $ & $\beta \, / \, °$ & Strahlenversatz $s  \, / \, \si{\centi\meter}$\\
    \midrule
    10 & 6,7494 $\pm$ 0,0859  & 0,334 $\pm$ 0,009\\
    20 & 13,384 $\pm$ 0,1661  & 0,693 $\pm$ 0,017\\
    30 & 19,767 $\pm$ 0,2291  & 1,103 $\pm$ 0,025\\
    40 & 25,783 $\pm$ 0,3437  & 1,596 $\pm$ 0,033\\
    50 & 31,226 $\pm$ 0,4010  & 2,20  $\pm$ 0,04\\
    60 & 35,867 $\pm$ 0,5156  & 2,95  $\pm$ 0,04\\
    70 & 39,476 $\pm$ 0,5729  & 3,849 $\pm$ 0,034\\
    \bottomrule
    \end{tabular}
\end{table}

\subsection{Prisma}
Das Prisma besteht aus Kronglas. Dieses hat einen Brechungsindex von $n_\text{Kron} = 1,51673$ \cite{Kronglas}.%Vorbereitungsaufgabe
Die Dispersion $\delta$ ist im allgemeinen gegeben durch
\begin{align}
    \delta =& ( \alpha_1 + \alpha_2 ) - ( \beta_1 + \beta_2 ) \\
    =& (\alpha_1+\alpha_2)- \left( \arcsin \left(\frac{\sin(\alpha_1)}{n_{kron}}\right) + \arcsin \left(\frac{\sin(\alpha_2)}{n_{kron}}\right) \right)\, ,
\end{align}

\noindent
wobei im allgemeinen die Winkelbeziehung $\beta_1 + \beta_2 = \gamma$ hergeleitet werden. Im Prisma gilt im allgemein $\gamma = 60°$.\\

\noindent
Bei der Messung wurde mit einem grünen Laser und einem roten Laser gemessen. Die Messergebnisse befinden sich in der \autoref{tab:las}. Mithilfe dieser und $\gamma = 60°$ lässt sich die Dispersion 
berechnen. Die berechneten Werte für diese sind in der \autoref{tab:dis} eingetragen. Zusätzlich lässt sich noch über den Brechungsindex des Prismas die Dispersion berechnen. Diese Messergebnisse 
befinden sich in der \autoref{tab:dis2}.

\begin{table}
    \centering
    \caption{Messwerte des Einfallswinkels $\alpha_1$ und den Austrittswinkel $\alpha_2$, je für das grüne und rote Laserlicht}
    \label{tab:las}
    \sisetup{table-format=2.1}
    \begin{tabular}{c c c}
    \toprule
         $\alpha_1 \, / \, ° $ & Rot $\alpha_2 \, / \, °$ & Grün $\alpha_2 \, / \, °$\\
    \midrule
    30 & 81,5 & 83,5 \\
    40 & 61   & 61,5 \\
    50 & 50   & 50,5 \\
    60 & 41,6 & 42,4 \\
    70 & 36   & 36,7 \\
    \bottomrule
    \end{tabular}
\end{table}

\begin{table}
    \centering
    \caption{Errechnete Werte für die Dispersion vom roten und grünen Laser}
    \label{tab:dis}
    \sisetup{table-format=2.1}
    \begin{tabular}{c c c}
    \toprule
         $\delta_\text{Rot} \, / \, ° $ & $\delta_\text{Grün} \, / \, °$\\
    \midrule
    51,5 &  53,5 \\
    41,0 &  41,5 \\
    40,0 &  40,5 \\
    41,6 &  42,4 \\
    46,0 &  46,7 \\ 
    \bottomrule
    \end{tabular}
\end{table}

\begin{table}
    \centering
    \caption{Errechnete Werte für die Dispersion vom roten und grünen Laser}
    \label{tab:dis2}
    \sisetup{table-format=2.1}
    \begin{tabular}{c c c}
    \toprule
         $\delta_\text{Rot} \, / \, ° $ & $\delta_\text{Grün} \, / \, °$\\
    \midrule
    51,55 &  53,33 \\
    40,71 &  41,02 \\
    39,32 &  39,58 \\
    40,82 &  41,19 \\
    44,91 &  45,21 \\ 
    \bottomrule
    \end{tabular}
\end{table}

\subsection{Beugung am Gitter}
Nun wird die Beugung der Laser am Gitter untersucht. Mithilfe des Zusammenhangs
\begin{equation}
    \lambda = d \frac{\sin \varphi}{k}
\end{equation}

kann bei bekannter Gitterkonstanten $d$, Ablenkwinkel $\varphi$ und Beugungsordnung $k$ die Wellenlänge berechnet werden. Die Messwerte der unterscheidlichen Gitter sind in der 
\autoref{tab:600}, \autoref{tab:300}, \autoref{tab:100}. Es wird für jeden Laser die errechneten Wellenlängen $\lambda$ gemittelt und die Standardabweichung mithilfe der Python
Bibliothek uncertainties berechnetund einmal für alle Gitter und komplett gemittelt in \autoref{tab:lambda} angegeben.

\begin{table}
	\centering
	\caption{Messwerte am Gitter mit 600 Linien/mm.}
	\label{tab:600}
	\sisetup{table-format=2.1}
	\begin{tabular}{c c c c c}
		\toprule
		Beugungsordnung $k$ &
		$\varphi_1^\text{rot}  \, / \, ° $  &
		$\varphi_2^\text{rot}  \, / \, ° $  &
		$\varphi_1^\text{grün} \, / \, ° $  &
		$\varphi_2^\text{grün} \, / \, ° $  \\
		\midrule
		1 & 22,5    & 23  &  19 & 19 \\
	\end{tabular}
\end{table}
\begin{table}
	\centering
	\caption{Messwerte am Gitter mit 300 Linien/mm.}
	\label{tab:300}
	\sisetup{table-format=2.1}
	\begin{tabular}{c c c c c}
		\toprule
		Beugungsordnung $k$ &
		$\varphi_1^\text{rot}  \, / \, ° $   &
		$\varphi_2^\text{rot}  \, / \, ° $   &
		$\varphi_1^\text{grün} \, / \, ° $  &
		$\varphi_2^\text{grün} \, / \, ° $  \\
		\midrule
		3 & 34,2    & 35  &  28,5   & 28,5 \\
		2 & 22 & 22,25   &  18,8 & 18,5 \\
        1 & 10,8 & 10,9   &  9,3 & 9 \\
		\bottomrule
	\end{tabular}
\end{table}
\begin{table}
	\centering
	\caption{Messwerte am Gitter mit 100 Linien/mm.}
	\label{tab:100}
	\sisetup{table-format=2.1}
	\begin{tabular}{c c c c c}
		\toprule
		Beugungsordnung $k$ &
		$\varphi_1^\text{rot}  \, / \, ° $   &
		$\varphi_2^\text{rot}  \, / \, ° $   &
		$\varphi_1^\text{grün} \, / \, ° $   &
		$\varphi_2^\text{grün} \, / \, ° $  \\
		\midrule
		5 & 18,7 & 18,9 & 15,9 & 16,5 \\
        4 & 14,8 & 15,1 & 12,5 & 12,4 \\
        3 & 11 & 11,2 & 9,4 & 9,2 \\
        2 & 7,2 & 7,5 & 6,3 & 6 \\
        1 & 3,5 & 3,9 & 3,1 & 3,1 \\
		\bottomrule
	\end{tabular}
\end{table}


\begin{table}
	\centering
	\caption{Errechnete Wellenlängen für den roten und grünen Laser.}
	\label{tab:lambda}
	\sisetup{table-format=2.1}
	\begin{tabular}{c c c c c}
		\toprule
		&
		$\lambda_\text{rot}$ &
		$\lambda_\text{grün}$ \\

		$100 \, \text{Linien} / \si{\milli\meter}$ &
		(643 \pm 17,2) \si{\nano\meter} &
		(542 \pm 11,1) \si{\nano\meter} \\
		$300 \, \text{Linien} / \si{\milli\meter}$ & 
		(622 \pm 4,7) \si{\nano\meter} &
		(525 \pm 5,6) \si{\nano\meter} \\
		$600 \, \text{Linien} / \si{\milli\meter}$ & 
		(645 \pm 6,7) \si{\nano\meter} &
		(543 \pm 0) \si{\nano\meter} \\
        Komplett gemittelt & 
		(637 \pm 6) \si{\nano\meter} &
		(537 \pm 4) \si{\nano\meter} \\
		\bottomrule
	\end{tabular}
\end{table}


