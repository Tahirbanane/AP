\section{Diskussion}

\subsection{Reflextionsgesetz}
Laut dem Reflextionsgesetz wird das Licht im einfallenden Winkel reflektiert. 
Bei der Auswertung fällt sofort auf das dies experimentell bestätigt wird. Trotzdessen
können während der Messungen aufgetreten sein. 

Da mit Unterlagen gearbeitet wurde besteht eine mögliche Fehlerquelle in dem nicht richtig 
sitzen diser. Eine andere besteht darin das die Skala an welcher abgelesen wurde nicht super
exakt ist.

\subsection{Brechungsgesetz}

In der Auswertung wurde der Brechungsindex von Plexiglas bestimmt zu
\begin{align*}
    n = 1.478 \pm 0.018 
\end{align*} 

und die Schallgeschwindigkeit 
\begin{align*}
    v = 2.029 \cdot 10^8 \, \pm \, 0.025 \cdot 10^8 \, \si{\meter\per\second} \, .
\end{align*}

Laut Literatur \cite{Kronglas} sind die Werte für den Brechungsindex und die Schallgeschwindigkeit
gegeben als

\begin{align*}
    n_\text{Lit} =& 1.491 \\
    v_\text{Lit} =& 2.01 \cdot 10^8 \, \si{\meter\per\second} \, .
\end{align*}

Mithilfe der Formel 
\begin{equation}
\label{eqn:o}
\Delta p = \bigg |\frac{f-g}{f} \bigg |,
\end{equation}

\noindent
wobei p die Prozentuale Abweichung darstellt und f und g jeweilige Zahlenwerte dessen Abweichungen zu einander untersucht werden sollen. So ergibt sich für die Abweichungen folgende Werte

\begin{align*}
    p_\text{index} =&  0,87 \si{\percent}   \\
    p_\text{geschwindigkeit} =& 0,945 \si{\percent}  \, .\\
\end{align*}

Die Abweichungen sind sehr gering und die Messung kann als Bestätigung angenommen werden.
Fehlerursachen können auch hier nicht richtig sitzende Messinstrumente sein, in diesem Fall die Skala,
sowie die Halterungen welche etwas spiel hattten.

\subsection{Gitter}
Licht mit der Farbe rot liegt normalerweise im Bereich von $ 650 \text{ bis } 750 \si{\nano\meter}$
und Licht mit der Farbe grün liegt normalerweise im Bereich von $490 \text{ bis } 575 \si{\nano\meter}$ \cite{}.

Experimentell wurde für die Laser jeweils eine Wellenlänge von 

\begin{align*}
\lambda_\text{rot} =& (637 \pm 6) \si{\nano\meter} \\
\lambda_\text{grün} =& (537 \pm 4) \si{\nano\meter} 
\end{align*}

bestimmt. Das Grüne Licht liegt genau in dem Intervall, beim roten Licht gibt es mit \autoref{eqn:o}
eine Abweichung von 
\begin{align*}
    p_\text{rot} = 2 \si{\percent}   
\end{align*}

zu $650 \, \si{\nano\meter}$. Diese Abweichung kann zum einen durch die nicht extrem genaue Skala
, einer wahrscheinlich minimal versetzten Unterlage, sowie
dem erklärt werden kann, dass es schwierig ist eine klare Grenze auszumachen, wann eine bestimmte Wellenlänge
eine bestimmte Farbe ist.
