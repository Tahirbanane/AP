\newpage
\section{Durchführung}
Der Versuchsaufbau besteht aus einer Glasplatte, auf der sich zwei übereinander gelegte 
Lasermodule im Halbkreis schieben lassen. Die Laser emittieren dabei grünes ($\lambda$ = 532 nm)
und rotes ($\lambda$ = 635 nm) Licht.
Auf der Platte befinden sich außerdem Befestigungsmöglichkeiten für diverse Objekte.
Es gibt zudem die Möglichkeit, Schablonen unter die Glasplatte zu legen sowie Schirme aufzustellen, um Winkel besser ablesen zu können.

\noindent
Zuerst wird ein Spiegel in der Mitte befestigt, um das Reflexionsgesetz zu untersuchen.
Mithilfe einer passenden Schablone können die Ausfallswinkel $\alpha_2$ für sieben Einfallswinkel $\alpha_1$ abgelesen werden.
Dabei wird der grüne Laser verwendet.
Die gemessenen Winkel befinden sich in Tabelle \ref{tab:ref}

\noindent
Für die Untersuchung des Brechungsgesetzes wird die planparallele Platte montiert und wieder der grüne Laser verwendet.
Unter Verwendung einer anderen Schablone kann der Brechungswinkel $\beta$ für sieben Einfallswinkel $\alpha$ bestimmt werden, 
die sich in Tabelle \ref{tab:bre} befinden. 

\noindent
Anschließend wird das Prisma platziert, und beide Laser eingeschaltet.
Die Einfallswinkel $\alpha_1$ sind für beide Laser gleich, jedoch werden für den grünen und roten Laser unterschiedliche Brechungswinkel gemessen, 
die in Tabelle \ref{tab:las} notiert sind.

\noindent
Zuletzt werden drei unterschiedliche Gitter (600, 300 und 100 $\frac{lines}{mm}$) nacheinander platziert und jeweils die Interferenzmuster des roten und grünen Lasers auf einen Schirm projeziert.
Es werden jeweils die Winkel der Intensitätsmaxima gemessen sowie die dazugehörige Ordnung notiert.
Die Messwerte befinden sich in Tabelle %\ref{tab:}.
