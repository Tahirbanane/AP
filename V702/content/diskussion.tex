\newpage
\section{Diskussion}
\label{sec:Diskussion}
\subsection{Literaturwert Vergleich}
Die in diesem Versuch bestimmten Halbwertszeiten 
\begin{align*}
    T_\text{Vanadium-52} =& 219 \pm 11 \si{\second}  \\
    T_\text{Vanadium-52,genauer} =& 225 \pm 22 \si{\second}  \\
    T_\text{Rhodium-104} =& 210 \pm 50 \si{\second} \\
    T_\text{Rhodium-104i} =& 48 \pm 1.2 \si{\second},
\end{align*}

\noindent
können auf ihre Genaugigkeit Mithilfe von Literaturwerten untersucht werden. Diese sind lauten laut \autocite{vanadium} und \autocite{rhodium}

\begin{align*}
    T_\text{Lit:Vanadium-52} =& 224,6 \si{\second}  \\
    T_\text{Lit:Rhodium-104} =& 260,4 \si{\second} \\
    T_\text{Lit:Rhodium-104i} =& 42,34  \si{\second}.
\end{align*}

\noindent 
Mithilfe der Formel 
\begin{equation*}
\Delta p = \bigg |\frac{f-g}{f} \bigg |,
\end{equation*}

\noindent
wobei p die Prozentuale Abweichung darstellt und f und g jeweilige Zahlenwerte dessen Abweichungen zu einander untersucht werden sollen. So ergibt sich für die Abweichungen folgende Werte

\begin{align*}
    p_\text{Vanadium-52} =&   2.493 \si{\percent}   \\
    p_\text{Vanadium-52,genauer} =& 0.178 \si{\percent}  \\
    p_\text{Rhodium-104} =&  19.355  \si{\percent}\\
    p_\text{Rhodium-104i} =&  13,368 \si{\percent}.
\end{align*}

\noindent
Es fällt auf, dass die Halbwertszeit von Vanadium-52 mindestens um eine Ordnung von $10^{1}$ genauer sind als die vom Rhodium-104. Dies lässt sich zum Teil darauf zurükführen,
dass beim Rhodium durch die zwei Zerfallsmöglichkeiten eine Abschätzung getroffen werden musste, in welchem Bereich lediglich der langsamere Zerfall stattfindet. Eine nicht exakt präzise
Wahl dieses Intervalls kann zu dem so großen Fehler geführt haben.
Es lässt sich daraus schlussfolgern, dass diese Art der Messung nicht besonders gut zur Bestimmung der Halbwertszeiten geeignet ist. Es wäre sinnvoller ein getrenntes Messgerät zu haben um die
Emissionen der $\gamma$-Quanten zu messen, um so besser auf den Zerfall rückschließen zu können.

\noindent
Zudem ist die Anzahl an Messpunkten für den schnellen Zerfall nicht ausreichend, da beim Vanadium-52 40 Messpunkte vorlagen und beim Rhodium-104i lediglich 10. Zudem kann eine weitere Fehlerquelle
darin bestehen, dass die Halbwertszeit von Rhodium-104i von der Halbwertszeit von Rhodium-104 abhängt. Dieser aht jedoch schon bereits einen relativ großen Fehler, welcher ebenfalls zu einem noch größerem
Fehler bei der Halbwertszeit von Rhodium-104i führt.

\subsection{sonstige Fehler}
Da der Versuch nicht von uns selbst durchgeführt wurde, können noch weitere Fehlerquellen vorliegen. Es ist möglich, dass nach der Aktivierung der Proben nicht schnell genug begonnen wurde zu messen, sodass
ein Großteil der Rhodium-104i Isomere bereits zerfallen sind. Zudem ist es denkbar das die Zeitintervalle beim Ablesen der Messwerte nicht exakt eingehalten wurden, sodass ebenfalls ein Fehler hier möglich ist.
Ebenfalls sind systematische Messfehler der Messapperatur denkbar, welche ebenfalls zu Fehlern führen könnten.

\noindent
Alle diese Faktoren führen dazu, dass die theoretischen Werten von praktisch gemessenen Werten abweichen und die Ergebnisse verfälscht wurden.