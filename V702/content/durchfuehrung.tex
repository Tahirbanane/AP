\section{Versuchsaufbau}

\begin{figure}
            \centering
               \includegraphics[height=5cm]{aufbau.pdf}
               \caption{Versuchsaufbau (Quelle: \cite{V702}).}
               \label{fig:aufbau}
\end{figure}

\noindent
Dieser Versuch wird mit einem Aufbau nach Abbildung (\ref{fig:aufbau}) durchgeführt.
Mit einem Geiger-Müller-Zählrohr, dessen Zählrohr sich in einem Abschirmblock aus Blei befindet um Umgebungsradioaktivität abzuschirmen,
können von zylindrischen Proben emittierte Teilchen nachgewiesen werden.
Die registrierten Teilchen liefern am Verstärkerausgang einen elektrischen Impuls,
der zum Zählwerk weitergeleitet wird.
Das Anzeigegerät kann periodisch zwischen beiden Zählern umschalten, 
wobei die das Zeitintervall $\Delta \text{t}$ mit dem Zeitgeber eingestellt wird.
Befindet sich auf dem Geiger-Müller-Zählrohr keine aktivierte Probe, wird der Nulleffekt gemessen.

\section{Durchführung}
Der Versuchsaufbau wird nach Abbildung (\ref{fig:aufbau}) aufgebaut.

\subsection{Aufgabe 1)}
\noindent
Zunächst wird der Untergrund bestimmt. 
Das Messintervall wird t=300s gewählt und mit einem Codierschalter an der Frontplatte des Gerätes eingestellt.
Die Untergrundrate wird mehrfach gemessen und notiert.
In Tabelle (\ref{tab:untergrundrate}) sind die gemessenen Werte aufgeführt.

\subsection{Aufgabe 2)}
\noindent
Zur Bestimmung der Halbwertszeit wird die Vanadiumprobe in der Neutronenquelle nach Abbilung (\ref{fig:nq}) 
aktiviert und 
danach direkt auf das Geiger-Müller-Zählrohr gesteckt.
Das Messintervall wird $\Delta \text{t} = 30 \si{\second}$ gewählt und die Messung wird gestartet.
Die Messwerte werden von der Anzeige abgelesen, notiert und sind in Tabelle (\ref{tab:vana}) aufgeführt.


\subsection{Aufgabe 3)}
Analog zur Halbwertszeitbestimmung von Valadium wird die Rhodiumprobe aktiviert und auf das Geiger-Müller-Zählrohr gesteckt,
wobei hier ein Messintervall von $\Delta \text{t} = 15 \si{\second}$ gewählt wird.
Die Messung wird gestartet und die Messwerte werden aufgenommen.
Diese sind in Tabelle (\ref{tab:rhodium}) aufgeführt.