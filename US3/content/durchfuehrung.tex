\section{Durchführung}
Für den Versuch stehen 3 Strömungsrohre verschiedener Innendurchmesser (7 mm, 10 mm und 16 mm) die mit Schläuchen verbunden sind zur Verfügung.
Durch das Rohrsystem fließt eine Dopplerflüssigkeit die durch eine Zentrifugalpumpe angetrieben wird.
Die Ankopplung der Ultraschallsonde an die Rohre erfolgt durch Ultraschall Gel und Doppler-Prismen, die auf die Rohre gesetzt werden
und drei Prismenwinkel haben, sodass eine reproduzierbare Durchführung und Einstellung des Dopplerwinkels garantiert ist.

\noindent
Um die Strömungsgeschwindigkeit zu bestimmen wird die Ultraschallsonde mit Ultraschallgel an eine Fläche des Prismas gehalten, 
welches sich auf einem der Strömungsrohre befindet.
Mithilfe des Programms FlowView kann der Wert f-mean bestimmt werden.
Die Flussgeschwindigkeit der Dopplerflüssigkeit wird an der Zentrifugalpumpe erhöht und der dazugehörige f-mean Wert wird notiert.
Dieser Vorgang wird für alle drei Prismenwinkel und alle drei Strömungsrohre durchgeführt.
Die dabei aufgenommenen Messwerte sind in Tabelle (\ref{tab:a}) notiert.

\noindent
Zur Bestimmung des Strömungsprofils wird die Ultraschallsonde mit Ultraschallgel bei einem Prismenwinkel von 15° an das Prisma gehalten, 
wobei nur das mittlere Strömungsrohr verwendet wird.
Nun kann am Ultraschallgenerator die Meßtiefe varriiert werden.
Die Messung wird für eine Tiefe von 13$\mu$s bis 17,5$\mu$s durchgeführt, 
während jeweils f-mean und der Streuintensitätswert notiert wird.
Dies wird für eine Flussgeschwindigkeit von 5,3 l/min und 3,4 l/min durchgeführt.
In Tabelle (\ref{tab:profil})
befinden sich die Messwerte.
