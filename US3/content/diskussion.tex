\section{Diskussion}
\label{sec:Diskussion}
\subsection{Beobachtungen}
Es fällt auf das bei allen drei Funktionen in den Abbildungen linear sind. Zudem ist zu beobachten, dass, wie erwartet, das Rohr mit dem kleinsten Durchmesser betragsmäßig die größte und das 
Rohr mit dem größten Durchmesser betragsmäßig die kleinste Geschwindigkeit aufweist. Die negativen Werte für die Geschwindigkeit kann durch die unterschiedlichen Messwinkel erklärt werden.

\noindent
Wenn die Skalen der beiden Plots bezüglich der Messintensitäten verglichen werden, fällt auf, dass diese sich relativ gering von einander unterscheiden. Zudem fällt auf, dass die Intensität
und Geschwindigkeit zur mitte des Rohres zunimmt und und zu den Rändern abfällt.

\noindent
Da alle erwarteten Phänomene beobachten wurden, kann geschlussfolgert werden, dass während des Versuchs keine groben Fehler gemacht wurden.

\subsection{Allgemeine Fehlerquellen}
\noindent
Während der Durchführung des Versuchs sind einige Fehlerquellen aufgefallen, die die Ergebnisse beeinflussen.

\noindent
Es muss darauf geachtet werden, dass genug Ultraschall Gel zwischen der Ultraschallsonde und dem Prisma, 
sowie zwischen dem Prisma und dem Strömungsrohr ist, da sonst die Werte schnell gegen 0 gehen und verfälscht werden können.

\noindent
Außerdem muss die Ultraschallsonde still gehalten werden, da die gemessenen Werte während der Messung stark schwanken.
Dabei muss darauf geachtet werden, dass die Sonde mittig auf den Flächen des Prismas aufliegt, da sie schnell verrutschen kann.
Dies ist bei der ersten Messung aufgefallen, da durch das Verrutschen die Werte für f-mean kleiner wurden, 
woraufhin die Messung wiederholt wurde.

\noindent
