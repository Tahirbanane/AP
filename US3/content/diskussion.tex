\section{Diskussion}
\label{sec:Diskussion}
Während der Durchführung des Versuchs sind einige Fehlerquellen aufgefallen, die die Ergebnisse beeinflussen.

\noindent
Es muss darauf geachtet werden, dass genug Ultraschall Gel zwischen der Ultraschallsonde und dem Prisma, 
sowie zwischen dem Prisma und dem Strömungsrohr ist, da sonst die Werte schnell gegen 0 gehen und verfälscht werden können.

\noindent
Außerdem muss die Ultraschallsonde still gehalten werden, da die gemessenen Werte während der Messung stark schwanken.
Dabei muss darauf geachtet werden, dass die Sonde mittig auf den Flächen des Prismas aufliegt, da sie schnell verrutschen kann.
Dies ist bei der ersten Messung aufgefallen, da durch das Verrutschen die Werte für f-mean kleiner wurden, 
woraufhin die Messung wiederholt wurde.

\noindent
