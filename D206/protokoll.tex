\documentclass[captions=tableheading]{scrartcl}
\usepackage[aux]{rerunfilecheck}
\usepackage{fontspec}
\usepackage[ngerman]{babel}
\renewcaptionname{ngerman}{\figurename}{Abb.}
\renewcaptionname{ngerman}{\tablename}{Tab.}
\usepackage{scrhack}
\usepackage{amsmath}
\usepackage{amssymb}
\usepackage{mathtools}
\usepackage[unicode]{hyperref}
\usepackage{bookmark}
\usepackage{booktabs}
\usepackage[section, below]{placeins}
\usepackage{caption}
\usepackage{graphicx}
\usepackage{subcaption}
\usepackage{float}

\usepackage[
    math-style=ISO,
    bold-style=ISO,
    sans-style=italic,
    nabla=upright,
    partial=upright,
]{unicode-math}

\usepackage[
  locale=DE,
  separate-uncertainty=true,
  per-mode=symbol-or-fraction,
]{siunitx}

\floatplacement{figure}{htbp}
\floatplacement{table}{htbp}
\usepackage[
  section,
  below,
]{placeins}
\setmathfont{Latin Modern Math}


\floatplacement{figure}{htbp}
\floatplacement{table}{htbp}


\usepackage{longtable}

\begin{document}
    \tableofcontents
    
    \newpage
    \section{Zielsetzung}
    Mit dem Versuch "`Wärmepumpe"' soll die Funktionsweise einer Wärmepumpe verstanden und zudem soll eine Aussage über die Güteziffer und den Masssendurchsatz,
    sowie den Wirkungsgrad des Kompressors getroffen werden.

    \section{Theoretische Grundlage}
    Beobachtungen und der zweite Hauptsatz der Thermodynamik zeigen, dass der Wärmefluss zwischen zwei gekoppelten Medien (z.B. durch ein Transportmedium), unterschiedlicher Temperatur, immer vom wärmeren zum kälterem zeigt.

    Es ist jedoch möglich diesen Wärmefluss umzudrehen. Dafür muss jedoch zusätzliche Energie aufgebracht werden zum Beispiel in Form von mechanischer. Eine Vorrichtung die in der Lage ist dies zu tun ist 
    eine sogenannte Wärmepumpe. Dazu nutzt es eines Kompressors und ein Transportmedium, sowie ein Ventil. 

        \subsection{Bestimmung der realen Güteziffer}
    	Das Verhältnis der transportierenden Wärmemenge und der dafür notwendigen Arbeit($A$) beschreibt die Größe der Güteziffer($\upsilon$) (unter idalisierten Bedingungen).
        Diese kann aus dem ersten Hauptsatz deer Thermodynamik hergeleitet werden:

        Der erste Hauptsatz der Thermodynamik besagt, dass Energien ineinander umwandelbar sind, aber nicht gebildet, bzw. vernichtet werden können. 
        Dies bedeutet im Kontext der Wärmepumpe, dass

        \begin{equation}
            Q_1 = Q_2 + A
            \label{eqn:Th1}
        \end{equation}

        gilt, wobei $Q_1$ die Wärme, welche von dem Transportmedium abgegeben wird und $Q_2$ die Wärme, welche an das Transportmedium abgegeben wird, darstellt. Daraus ergibt sich für die Güteziffer:
        \begin{equation}
            \upsilon = \frac{Q_\text{transp}}{A} \stackrel{(\ref{eqn:irev})}{\implies} \upsilon_\text{id} = \frac{T_1}{T_1 - T_2}.
            \label{eqn:Gueteziffer}
        \end{equation}

        Aus dem zweiten Hauptsatz der Thermodynamik und der Annahme, dass während der Wärmeübertragung kein Wärmeverlust der beiden Resoervoire stattfindet
         und die Wärmeübertragung reversibel verläuft,ergibt sich idialisiert:

        \begin{equation}
            \frac{Q_1}{T_1} - \frac{Q_2}{T_2} = 0.
            \label{eqn:irev}
        \end{equation}
        
        Da jedoch eine Wärmepumpe in der Realität nicht in der Lage ist den Prozess der Wärmeübertragung vollständig reversibel durchzuführen und dieser dardurch irreversibel wird, gilt:

        \begin{equation}
            \frac{Q_1}{T_1} - \frac{Q_2}{T_2} > 0.
            \label{eqn:rev}
        \end{equation}

        So ergibt sich für eine reale Wärmepumpe mit \ref{eqn:Th1} und \ref{eqn:rev}

        \begin{equation}
            \upsilon_\text{id} < \frac{T_1}{T_1 - T_2}.
            \label{eqn:Gueteziffer_real}
        \end{equation}

        




        \subsection{Bestimmung des Massendurchsatzes}

        \subsection{Bestimmung der mechanischen Kompressorleistung $N_\text{mech}$}

    \newpage
    \section{Durchführung: ebenfalls}

    \newpage
    \section{Messwerte}
    Die folgenden Messwerte wurden uns zur Verfügung gestellt:
    \input{table}

    \newpage
    \section{Auswertung}
        \subsection{Aufgabenteil a)}
        \begin{figure}
               \centering
               \includegraphics[width=\textwidth]{grafic.pdf}
               \caption{Auswertung mit ausgleichsgeraden}
               \label{fig:grafic}
        \end{figure}


            

        \subsection{Aufgabenteil b)}


        \subsection{Aufgabenteil c)}


        \subsection{Aufgabenteil d)}


        \subsection{Aufgabenteil e)}


        \subsection{Aufgabenteil f)}


        \subsection{Aufgabenteil g)}
\end{document}