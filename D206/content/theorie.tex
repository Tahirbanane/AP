\section{Zielsetzung}
    Mit dem Versuch "`Wärmepumpe"' soll die Funktionsweise einer Wärmepumpe verstanden und zudem soll eine Aussage über die Güteziffer und den Massendurchsatz,
    sowie den Wirkungsgrad des Kompressors getroffen werden.

    \section{Theoretische Grundlage}
    Beobachtungen und der zweite Hauptsatz der Thermodynamik zeigen, dass der Wärmefluss zwischen zwei gekoppelten Medien (z.B. durch ein Transportmedium), unterschiedlicher Temperatur, immer vom wärmeren zum kälterem zeigt.

    Es ist jedoch möglich diesen Wärmefluss umzudrehen. Dafür muss jedoch zusätzliche Energie aufgebracht werden zum Beispiel in Form von mechanischer. Eine Vorrichtung die in der Lage ist dies zu tun ist 
    eine sogenannte Wärmepumpe. Dazu nutzt es eines Kompressors und ein Transportmedium, sowie ein Ventil. 

        %2.1
        \subsection{Bestimmung der realen Güteziffer}
    	Das Verhältnis der transportierenden Wärmemenge und der dafür notwendigen Arbeit($A$) beschreibt die Größe der Güteziffer($\upsilon$) (unter idealisierten Bedingungen).
        Diese kann aus dem ersten Hauptsatz der Thermodynamik hergeleitet werden:

        Der erste Hauptsatz der Thermodynamik besagt, dass Energien ineinander umwandelbar sind, aber nicht gebildet, bzw. vernichtet werden können. 
        Dies bedeutet im Kontext der Wärmepumpe, dass

        \begin{equation}
            Q_1 = Q_2 + A
            \label{eqn:Th1}
        \end{equation}

        gilt, wobei $Q_1$ die Wärme, welche von dem Transportmedium abgegeben wird und $Q_2$ die Wärme, welche an das Transportmedium abgegeben wird, darstellt. Daraus ergibt sich für die Güteziffer:
        \begin{equation}
            \upsilon = \frac{Q_\text{transp}}{A} \stackrel{(\ref{eqn:irev})}{\implies} \upsilon_\text{id} = \frac{T_1}{T_1 - T_2}.
            \label{eqn:Gueteziffer}
        \end{equation}

        Aus dem zweiten Hauptsatz der Thermodynamik und der Annahme, dass während der Wärmeübertragung kein Wärmeverlust der beiden Reservoire stattfindet
         und die Wärmeübertragung reversibel verläuft, ergibt sich idealisiert:

        \begin{equation}
            \frac{Q_1}{T_1} - \frac{Q_2}{T_2} = 0.
            \label{eqn:irev}
        \end{equation}
        
        Da jedoch eine Wärmepumpe in der Realität nicht in der Lage ist den Prozess der Wärmeübertragung vollständig reversibel durchzuführen und dieser Prozess dardurch irreversibel wird, gilt:

        \begin{equation}
            \frac{Q_1}{T_1} - \frac{Q_2}{T_2} > 0.
            \label{eqn:rev}
        \end{equation}

        So ergibt sich für eine reale Wärmepumpe mit (\ref{eqn:Th1}) und (\ref{eqn:rev}):

        \begin{equation}
            \upsilon_\text{real} < \frac{T_1}{T_1 - T_2}.
            \label{eqn:Gueteziffer_real}
        \end{equation}

        Aus dieser Gleichung (\ref{eqn:Gueteziffer_real}) folgt, dass die Wärmepumpe umso weniger Arbeit braucht, desto geringer die Differenz der beiden Temperaturen ist.

        Des weiteren lässt sich die pro Zeiteinheit gewonnene Wärmemenge $\frac{\Delta Q_1}{\Delta t}$ errechnen, 
        indem aus einer Messreihe $T_1$ als Funktion der Zeit t die Größe $\frac{\Delta T_1}{\Delta t}$ für ein geeignet gewähltes Zeitintervall $\Delta t$ ermittelt wird.
        Daraus ergibt sich dann 

        \begin{equation}
        \frac{\Delta Q_1}{\Delta t} = (m_1 c_\text{w} + m_\text{k} c_\text{k}) \frac{\Delta T_1}{\Delta t},
        \label{gueteziffer2}
        \end{equation}

        wobei $m_1 c_\text{w}$ die Wärmekapazität des Wassers im Reservoir 1 und $m_\text{k} c_\text{k}$ die Wärmekapazität der Kupferschlange und des Eimers bedeuten.
        Für die Güteziffer $\upsilon$ ergibt sich dann

        \begin{equation}
        \upsilon = \frac{\Delta Q_1}{\Delta t N}
        \label{eqn:gueteziffer3}
        \end{equation}

        mit N als die vom Wattmeter angezeigte und über das Zeitintervall $\Delta t$ gemittelte Kompressorleistung.


        %2.2
        \subsection{Bestimmung des Massendurchsatzes}

        Der Massendurchsatz berechnet sich nach [D206 verlinken, S. 5] über den Differenzquotienten über:
        \begin{equation}
            \frac{\Delta Q_2}{\Delta t} = \left( m_2  c_\text{w} + m_\text{k} c\text{k}\right) \frac{\Delta T_2}{\Delta t}
            \label{eqn:mass1}
        \end{equation}

        und 

        \begin{equation}
            \frac{\Delta Q_2}{\Delta t} = L \frac{\Delta m}{\Delta t}
            \label{eqn:mass2}
        \end{equation}

        mit (\ref{eqn:mass1}) und (\ref{eqn:mass2}) folgt:

        \begin{equation}
            L \frac{\Delta m}{\Delta t} = \left( m_2  c_\text{w} + m_\text{k} c\text{k}\right) \frac{\Delta T_2}{\Delta t} \iff \frac{\Delta m}{\Delta t} = \left( m_2  c_\text{w} + m_\text{k} c\text{k}\right) \frac{1}{L} \frac{\Delta T_2}{\Delta t},
        \end{equation}
        
        falls die Verdampfungswärme $L$ bekannt ist.


        %2.3
        \subsection{Bestimmung der mechanischen Kompressorleistung $N_\text{mech}$}

        Für die Arbeit $A_\text{m}$ des Kompressors, wenn er das Gasvolumen $V_\text{a}$ auf den Wert $V_\text{b}$ verringert, gilt:

        \begin{equation}
            A_\text{m} = - \int_{V_\text{a}}^{V_\text{b}} p \, \symup{d}V.
            \label{eqn:nmech1}
        \end{equation}

Näherungsweise wird nun angenommen, dass die Kompression adiabatisch erfolgt.
Für den Zusammenhang zwischen Druck und Volumen gilt die Poissonsche Gleichung

\begin{equation}
p_\text{a}V^\kappa_\text{a} = p_\text{b}V^\kappa_\text{b} = pV^\kappa.
\label{eqn:poisson}
\end{equation}
Damit erhält man für $A_\text{m}$
\begin{equation}
A_\text{m} = - p_\text{a}V^\kappa_\text{a}\int_{V_\text{a}}^{V_\text{b}} V^{-\kappa} \, \symup{d}V = \frac{1}{\kappa-1}p_\text{a}V^\kappa_\text{a}\Bigl(V^{-\kappa+1}_\text{b}-V^{-\kappa+1}_\text{a}\Bigr) = \frac{1}{\kappa-1}\biggl(p_\text{b}\sqrt[\kappa]{\frac{p_\text{a}}{p_\text{b}}} - p_\text{a}\biggr) V_\text{a}
\label{eqn:A_m}
\end{equation}

und für die mechanische Kompressorleistung $N_\text{mech}$
\begin{equation}
N_\text{mech} = \frac{\Delta A_\text{m}}{\Delta t} = \frac{1}{\kappa-1}\biggl(p_\text{b}\sqrt[\kappa]{\frac{p_\text{a}}{p_\text{b}}} - p_\text{a}\biggr) \frac{\Delta V_\text{a}}{\Delta t} = \frac{1}{\kappa-1}\biggl(p_\text{b}\sqrt[\kappa]{\frac{p_\text{a}}{p_\text{b}}} - p_\text{a}\biggr) \frac{1}{\rho} \frac{\Delta m}{\Delta t}.
\label{nmech2}
\end{equation}

$\rho$ ist dabei die Dichte des Transportmediums im gasförmigen Zustand beim Druck $p_\text{a}$.
Das hier benutzte Transportgas ist Dichlordiflourmethan ($\ce{Cl2F2C}$) mit $\rho_0 = 5,51\frac{g}{l}$. 
Mit Hilfe der idealen Gasgleichung kann nun $\rho$ für Normalbedingungen (p = 1 Bar, T = $\SI{0}{\celsius}$) berrechnet werden.

