\section{Diskussion}
\label{sec:Diskussion}
An eine ideale Wärmepumpe wird die Forderung gestellt, dass die Wärmeübertragung reversibel verlaufen muss.
Da das in der Realität nicht möglich ist, kann die vom Transportmedium aufgenommene Wärme 
und die mechanische Energie nicht jederzeit in einem umgekehrt laufenden Prozess vollständig wieder zurückgewonnen werden.
Des weiteren ist zu erwähnen, dass durch die Isolierung der Wärmeaustausch mit der Umgebung nicht vollständig verhindert wird. 
Somit erfolgt auch die Kompression nicht vollständig adiabatisch. 
Diese Faktoren wirken sich alle negativ auf die Güteziffer aus.

Da wir den Versuch nicht selber durchführen konnten, können wir keinerlei Urteil über die Empfindlichkeit der Gerätschaften treffen.
Es ist dennoch denkbar, dass der gesamte Aufbau durch äußere Einwirkungen beeinflusst werden und dies zu Leistungsabfällen führen kann.
Auffällig ist außerdem der Knick der Messwerte von $T_2$ in Abbildung 3, der durch Fehler beim ablesen oder Messungenauigkeiten verursacht worden sein könnte.
