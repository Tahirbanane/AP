\section{Diskussion}
\label{sec:Diskussion}
Während der Durchführung des Versuchs sind einige Fehlerquellen aufgefallen, die die Ergebnisse beeinflussen.

\noindent
Die Zählrate N stieg schon bei kleinen Erschütterungen am Tisch stark an, sodass praktisch mehr Zählraten gemessen wurden.
Dies fiel vorallem beim Erzeugen des Vakuums auf, da der Aufbau durch die Pumpe und den Schlauch leicht gewackelt hat.

\noindent
Eine weiterer beeinflussender Faktor ist die Ungenauigkeit beim Ablesen von Messgrößen wie dem Wegunterschied $\Delta \text{d}$ von der Nanometerschraube 
und dem Druckunterschied $\Delta \text{p}$, wo die Anzeignadel der Pumpe immer relativ schnell ausgeschlagen hat.

\noindent
Der berechnete Mittelwert der Wellenlänge des Lasers $\lambda = 635.9698 \si{\nano\meter}$ liegt mit einem relativen Fehler von  $\Delta_\text{rel}\lambda = 0.27\%$ nah an dem tatsächlichen Wert von $\lambda_\text{lit} = 635 \si{\nano\meter}$.


\noindent
Der berechnete Mittelwert des Bruchungsindexes von Luft $\text{n} = 1.000232$ liegt mit einem relativen Fehler von $\Delta_\text{rel} \text{n} = 0.006\%$ ebenfalls sehr nah am Literaturwert $\text{n}_\text{lit} = 1.000929$.

\noindent
Dabei liegen beide Literaturwerte jeweils im Bereich der Standardabweichung.



