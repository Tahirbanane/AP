\section{Zielsetzung}
    Ziel dieses Versuches ist es die Wellenlänge $\lambda$ und den Brechungsindex $n$ von Luft mithilfe des Michelson-Interferometers zu bestimmen.

\section{Theoretische Grundlage}
\noindent
\subsection{allgemeine Grundlagen}
Licht ist eine elektromagnetische Welle. Eine ebene Welle in ihrer einfachsten Form lässt sich durch ihre elektische Feldstärke in Orts- und Zeitabhängigkeit darstestellen als
\begin{equation}
    \vec{E}(x, t) = \vec{E_0} \cdot \text{cos}(kx - \omega t - \delta) \;.
    \label{eqn:ansatz}
    \caption{(x = Ortskoordinate, t = Zeit, k = Wellenzahl = 2π/λ, λ = Wellenlänge, ω = Kreisfrequenz, δ = Phasenwinkel in Bezug auf einen festen Zeit- und Ortsnullpunkt)}
\end{equation}
\noindent
Für Licht gilt im allgemeinen das Spuerpositionsprinzip. Dies bedeutet, dass das Elektrische Feld als eine Komposition aus vielen anderen Feldstärken dargestellt werden kann.
Da die Feldstärken des Lichtes auf Grund der hohen Frequenz, in der größen Ordnung $10^(15) \si{\hertz}$, nicht messen kann, kann diese aus der Intensität, mit dem Zusammenhang
\begin{equation}
    \text{I} = \text{const} |x|^2 \; ,
    \label{eqn:Int}
\end{equation}
\noindent
welche aus den Mexwell-Gleichung, berechnet werden. Für die Intensität gilt jedoch nicht das Spuerpositionsprinzip in klassischen Sinne. Falls die Wellen $\vec{E_1}$ und $\vec{E_2}$ 
an einem Ort $x$  aufeinander treffen, ergibt sich unter der Bedingung, dass $t_2 - t_1$ groß gegen die Periodendauer $T = \frac{2 \pi}{\omega}$ ist, als

\begin{equation}
    I_\text{ges} = \frac{C}{t_2 - t_1} \int^{t_2}_{t_1} |\vec{E_1} + \vec{E_2}|^2 (x, t) \: \text{d}t 
    \; , \; C \text{ konstant, } 
\end{equation}

\subsection{Kohärenz}