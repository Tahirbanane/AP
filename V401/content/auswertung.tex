\newpage
\section{Auswertung}
Die Wellenlänge $\lambda$ kann mit der Formel () berrechnet werden, 
wobei die Hebelübersetzung mit $\text{Ü}=5,046$ berücksichtigt werden muss.
Somit ergibt sich die Formel zu

\begin{equation}
\lambda=\frac{2 \Delta \text{d}}{ \text{NÜ}} .
\label{eqn:lam}
\end{equation}

\noindent
Die aufgenommenen Messwerte und die daraus nach (\ref{tab:lam}) berechneten Wellenlängen des Lasers sind in Tabelle (\ref{tab:lambda}) aufgeführt.

\begin{table}
    \centering
    \caption{Messwerte und die berechneten Wellenlängen}
    \label{tab:lam}
    \sisetup{table-format=2.1}
    \begin{tabular}{c c c}
    \toprule
    $ \Delta \text{d} \,/\, \si{\milli\meter} $ & N & $\lambda \,/\, \si{\nano\meter}$\\
    \midrule 
   
    4.835  & 2743 &  698.639 \\
    5.060  & 3149 &  636.884 \\
    4.550  & 2855 &  631.667 \\
    5.005  & 3170 &  625.788 \\
    4.950  & 3134 &  626.021 \\
    4.955  & 3136 &  626.254 \\
    5.000  & 3133 &  632.546 \\
    5.010  & 3169 &  626.611 \\
    5.000  & 3149 &  629.332 \\
    5.000  & 3166 &  625.953 \\

    \bottomrule
    \end{tabular}
  \end{table}

\noindent
Daraus ergibt sich der Mittelwert und die Standardabweichung der Wellenlänge

\begin{equation*}
\lambda = (635.9698 \pm 22.3299) \si{\nano\meter}.
\end{equation*}

\noindent
Der Brechungsindex von Luft kann nach Gleichung () berechnet werden.
Es gilt:

\begin{align*}
\text{p}_0 &= 1.0132 \si{\bar}\\
\text{T}_0 &= 273.15 \si{\kelvin}\\
\text{T} &= 293.15 \si{\kelvin}\\
\text{b} &= 50 \si{\milli\meter}
\end{align*}

\noindent
Die aufgenommenen Messwerte und die daraus nach () berechneten Brechungsindizes von Luft sind in Tabelle (\ref{tab:n}) aufgeführt.

\newpage
\begin{table}
  \centering
  \caption{Messwerte und die berechneten Brechungsindizes}
  \label{tab:n}
  \sisetup{table-format=2.1}
  \begin{tabular}{c c c}
  \toprule
  $ \Delta \text{p}$ \,/\, $\text{torr} $ & N & $\text{n}$\\
  \midrule 
 
    610 & 27 & 1.000229 \\
    610 & 36 & 1.000306 \\
    610 & 17 & 1.000145 \\
    610 & 35 & 1.000298 \\
    610 & 18 & 1.000153 \\
    610 & 34 & 1.000289 \\
    610 & 18 & 1.000153 \\
    600 & 33 & 1.000285 \\

  \bottomrule
  \end{tabular}
\end{table}

\noindent
Daraus ergibt sich der Mittelwert und die Standardabweichung des Brechungsindexes

\begin{equation*}
\text{n} = 1.000232 \pm 0.000072.
\end{equation*}
