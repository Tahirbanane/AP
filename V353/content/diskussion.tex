\section{Diskussion}
\label{sec:Diskussion}
Die durch unterschiedliche Verfahren ermittelten Werte für den gleichen RC-Kreis liegen sehr dicht bei einander. 

\begin{align*}
    RC_\text{4.1} &= 0.0010262 \pm 0.0000517 \si{\second}\\
    RC_\text{4.2} &= 0.00153 \pm 0.00015 \si{\second}\\
    RC_\text{4.3} &= 0.0009829 \pm 0.00095 \si{\second}\\
\end{align*}

\noindent
Da leider die Werte aus dem RC-Kreis uns nicht bekannt sind, können wir leider
nicht auf die Genauigkeit und das beste Verfahren zur Ermittlung des RC Wertes eines RC-Kreises rückschließen. Da jedoch alle Werte relativ dicht bei einander liegen, kann angenommen werden,
dass alle Verfahren ungefähr gleich gut sind. 
\noindent
Die Abweichungen der Werte können untereinander können auf Messungenaugkeiten zurückgeführt werden, da zum Beispiel die Amplituden an einem digitalen Oszilloskop lediglich mit dem einem Kursor 
bestimmt wurden, welcher nicht unendlich genau eignestellt werden konnte. Zudem wurde nur mit Werten bis zu 2. Stelle nach dem Komma gerechnet, sodasss auch dardurch Ungenauigkeiten nicht auszuschließen sind.
Zudem kann ein schwanken in der Spannungsquelle, sowie ein Wackelkontakt im System für einen Fehler sorgen.