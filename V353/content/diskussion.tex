\section{Diskussion}
\label{sec:Diskussion}
Die durch unterschiedliche Verfahren ermittelten Werte für den gleichen RC-Kreis liegen sehr dicht bei einander. 

\begin{align*}
    RC_\text{4.1} &= 0.0010262 \pm 0.0000517 \si{\second}\\
    RC_\text{4.2} &= 0.00153 \pm 0.00015 \si{\second}\\
    RC_\text{4.3} &= 0.0009829 \pm 0.00095 \si{\second}\\
\end{align*}

\noindent
Da leider die Werte aus dem RC-Kreis uns nicht bekannt sind, können wir leider
nicht auf die Genauigkeit und das beste Verfahren zur Ermittlung des RC Wertes eines RC-Kreises rückschließen. Jedoch ist es trotzdem möglich die Abweichungen untereinander zu berechnen,
dies kann man mit:

\begin{equation*}
\Delta p = \frac{f-g}{f}
\end{equation*}
\noindent
berechnen, wobei man f und g die zu untersuchenden Zahlen sind und $\Delta p$ die Abweichung von f zu g angibt.

\begin{align*}
    \Delta p_\text{4.1. --> 4.2.} &= 49.09\% \\
    \Delta p_\text{4.1. --> 4.3.} &= 04.22\% \\
    \Delta p_\text{4.2. --> 4.3.} &= 35.76\% \\
\end{align*}


\noindent
Es fällt auf, dass der RC Wert aus 4.1. und 4.3. sehr dicht bei einander liegen, aber doch der Wert aus 4.2. stark abweicht. Daraus könnte geschlussfolgert werden, dass das Verfahren aus
4.2. ungenauer ist als beide anderen.
\noindent
Die Abweichungen der Werte untereinander können auf Messungenaugkeiten zurückgeführt werden, da zum Beispiel die Amplituden an einem digitalen Oszilloskop lediglich mit dem einem Kursor 
bestimmt wurden, welcher nicht unendlich genau eignestellt werden konnte. Zudem wurde nur mit Werten bis zu 2. Stelle nach dem Komma gerechnet, sodasss auch dardurch Ungenauigkeiten nicht auszuschließen sind.
Zudem kann ein schwanken in der Spannungsquelle, sowie ein Wackelkontakt im System für einen Fehler sorgen.