\newpage
\section{Diskussion}
\label{sec:Diskussion}
\noindent
Bei der Untersuchung des Selektivverstärkers ist aufgefallen, 
dass die Frequenz nicht eindeutig ablesbar ist, da sie von allein anstieg.
Die Messung wurde des Weiteren durch die Skalierung bis 200 kHz erschwert, da die Frequenz nur sehr unpräzise eingestellt werden konnte.
Trotzdem zeigt die Filterkurve aus Abbildung (\ref{fig:selektiv}) keine Auffälligkeiten und entspricht den Erwartungen.


\noindent
Nun sollen die Suszeptibilitäten betrachtet werden.
Um die Werte zu vergleichen, werden die Suszeptibilitäten und relativen Abweichungen vom theoretischen Wert in Tabelle (\ref{tab:disk}) notiert.

\begin{table}
    \centering
    \begin{tabular}{c c c c c c}
        \toprule
        {Stoff} & {$\chi_U $} & {$\chi_R $} & {$\chi_T$} & {$\Delta_{rel}\chi_U$} & {$\Delta_{rel}\chi_R$} \\
    \midrule
    $\ce{Dy2O3}$ & 0.1730 $\pm$ 0.0026  & 0.0208 $\pm$ 0.0001 & 0.0254 & 581\% & 40\% \\
    $\ce{Gd2O3}$ & 0.0785 $\pm$ 0.0038  & 0.0109 $\pm$ 0.0005 & 0.0149 & 209\% & 27\% \\
    \bottomrule
\end{tabular}
\caption{Suszeptibilitäten und relative Abweichungen.}
\label{tab:disk}
\end{table}

\noindent
Direkt fällt auf, dass die Abweichung von $\chi_U$ zum theoretischen Wert viel größer ist als die Abweichung von $\chi_R$.
Das könnte daran liegen, dass die Formel %chi_u
für $\chi_U$ nach Bedingung nur für hinreichend große Messfrequenzen gilt und das hier nicht erfüllt ist.

\noindent
Außerdem ist auch die Abweichung von $\chi_R$ bei beiden Stoffen sehr hoch.
Ein Grund dafür ist die Temperaturabhängigkeit der Suszeptibilität.
Wenn eine Probe zu lang in der Hand gehalten wird, kann dies die Messergebnisse beeinflussen.

\noindent
Eine weitere Fehlerquelle stellt die Vermessung der Proben dar, denn die Länge $l$ aus Tabelle (\ref{tab:proben}) kann nur
hinreichend genau bestimmt werden, da die staubförmigen Proben Löcher aufweisen, deren Länge nur grob bestimmt werden kann.

\noindent
Weiterhin sind auch Ablesefehler der Spannungen aus Tabelle (\ref{tab:exp1}) und (\ref{tab:exp2}) zu berücksichtigen, da das Millivoltmeter sehr empfindlich ist und leicht ausschlägt.

\noindent
Imsgesamt ist die Messung sehr fehleranfällig und die Ergebnisse wenig aussagekräftig.


