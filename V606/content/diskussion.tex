\newpage
\section{Diskussion}
\label{sec:Diskussion}
\noindent
Zunächst wird die Güte des Selektivverstärkers diskutiert.
Der Kurve aus Abbildung (\ref{fig:guete}) ist eine Güte von $ Q = 62.376 \pm 0.018 $ zu entnehmen.
Damit beträgt die relative Abweichung $\Delta_{rel} Q = 38\%$ von der gegebenen Güte $Q=100$.
Bei der Untersuchung des Selektivverstärkers ist aufgefallen, 
dass die Frequenz nicht eindeutig ablesbar ist, da sie von allein anstieg.
Die Messung wurde des Weiteren durch die Skalierung bis 200 kHz erschwert, da die Frequenz nur sehr unpräzise eingestellt werden konnte.
Außerdem ist anzumerken, dass auf dem Selektivverstärker "defekt" bei der Güte 100 stand,
wodurch eine geringere Güte zu erwarten war.

\noindent
Nun sollen die Suszeptibilitäten betrachtet werden.
Um die Werte zu vergleichen, werden die Suszeptibilitäten und relativen Abweichungen vom theoretischen Wert in Tabelle (\ref{tab:disk}) notiert.

\begin{table}
    \centering
    \caption{Suszeptibilitäten und relative Abweichungen.}
    \begin{tabular}{c c c c c c}
        \toprule
        {Stoff} & {$\chi_U $} & {$\chi_R $} & {$\chi_T$} & {$\Delta_{rel}\chi_U$} & {$\Delta_{rel}\chi_R$} \\
    \midrule
    $\ce{Dy2O3}$ & 0.0173 $\pm$ 0.0003  & 0.0208 $\pm$ 0.0001 & 0.0254 & 32\% & 40\% \\
    $\ce{Gd2O3}$ & 0.0079 $\pm$ 0.0004  & 0.0109 $\pm$ 0.0005 & 0.0149 & 69\% & 27\% \\
    \bottomrule
\end{tabular}

\label{tab:disk}
\end{table}

\noindent
Beide Arten der experimentellen Bestimmung der Suszeptibilität weisen hohe Abweichungen zum theoretischen Wert auf. 
Bei $\chi_U$ kann ein Grund sein, dass die Formel (\ref{eqn:spannung}) nach Bedingung nur für hinreichend große Messfrequenzen gilt und das hier nicht erfüllt ist.

\noindent
Außerdem ist auch die Abweichung von $\chi_R$ bei beiden Stoffen hoch.
Ein Grund dafür ist die Temperaturabhängigkeit der Suszeptibilität.
Wenn eine Probe zu lang in der Hand gehalten wird, kann dies die Messergebnisse beeinflussen.

\noindent
Eine weitere Fehlerquelle stellt die Vermessung der Proben dar, denn die Länge $l$ aus Tabelle (\ref{tab:proben}) kann nur
hinreichend genau bestimmt werden, da die staubförmigen Proben Löcher aufweisen, deren Länge nur grob bestimmt werden kann.

\noindent
Weiterhin sind auch Ablesefehler der Spannungen aus Tabelle (\ref{tab:exp1}) und (\ref{tab:exp2}) zu berücksichtigen, da das Millivoltmeter sehr empfindlich ist und leicht ausschlägt.

\noindent
Imsgesamt ist die Messung sehr fehleranfällig und die Ergebnisse wenig aussagekräftig.


