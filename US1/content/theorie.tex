\section*{Zielsetzung}
In disem Versuch soll die Prinzipelle Funktionsweise eines Ultraschallgeräts untersucht und verstanden werden, sowie die Schallgeschwindigkeit in Acryl bestimmt werden.

\section{Theorie}
Im allgemeinen lässt sich die Frequenz mit der sich der Schall ausbreitet in 4 unterscheidliche Kathegorien unterteilen. Im Bereich von $f < \SI{16}{Hz}$ ist die Schallwelle als Infraschall,
im Bereich von $\SI{16}{Hz} < f < \SI{20}{kHz}$ als hörbarer Schall, im Bereich von $\SI{20}{kHz} < f < \SI{1}{GHz}$ als Ultraschall und alles mit $f > \SI{1}{GHz}$ als Hyperschall
klassifizieren. \\ 
  
Im allgemeinen lassen sich longitudinal ausbreitende Wellen, wie Schallwellen in Luft, beschreiben durch die Form
\begin{equation}
  p(x,t) = p_0 + v_0 Z \cos{(\omega t - kx)}
\end{equation}
\\
beschreiben, denn diese Wellen sind lediglich Druckschankungen in der Luft. 
Schallwellen breiten sich in Festkörpern jedoch nicht nur longitudinal, sondern auch transversal. (Es entstehen somit keine Druck unterschiede, sondern Schubspannungen). 
Dementsprechend ist die Phasengeschwindigkeit druck-, dichte- und somit
materialabhängig. \\
Die Schallgeschwindigkeit in Flüssigkeiten lässt sich beschreiben durch 
\begin{equation}
  c_\text{Fl} = \sqrt{\frac{1}{\kappa \rho}}
  \label{eqn:schallgeschwindigkeit_flussigkeit}
\end{equation}

und in Festkörpern durch 
\begin{equation}
  c_\text{Fe} = \sqrt{\frac{E}{\rho}}
  \label{eqn:schallgeschwindigkeit_festkörper}
\end{equation}

, wobei $\kappa$ die Kompressibilität und  $E = \frac{1}{\kappa}$ Elastizitätsmodul beschreibt.

\subsection{Absorbtions und Reflektion}
Schallwellen werden an Grenzenflächen zwischen Festkörpern und Gasen oder Festkörpern zum einen Teil reflektiert und transmetiert. 
Der Reflektionskoeffizient ist im allgemeinen gegeben durch
\begin{equation}
  R = \left(\frac{Z_1 - Z_2}{Z_1 + Z_2}\right)^2 \ ,
  \label{eqn:reflexion}
\end{equation}

wobei $Z= \rho \cdot c$ für die akustische Impedanz der beiden Stoffe steht.
Damit lässt sich der Transmissionskoeffizient bestimmen zu
\begin{equation}
  T = 1 - R \ .
  \label{eqn:transmission}
\end{equation}


Durch die Ausbreitung der Welle im Medium wird ein Teil dieser absorbiert. Dies kann durch die Intensität
\begin{equation}
	\label{eqn:daempfung}
	I(x) = I_0 \cdot e^{-\alpha x}
\end{equation}

wobei x die die Strecke der Welle und $\alpha$ der Absorptionskoeffizienten ist.


\subsection{Erzeugen von Ultraschall}

Ultraschall wird in der Regel mithilfe eines piezoelektrischer Kristalles erzeugt wird. Dazu wird dieser in ein oszillierendes elektisches Wechselfeld gebracht, wodurch dieser zur
Schwingung angeregt wird. Bei der Anregung mit ungefähr der Eigenfrequenz sind sehr große Amplituden möglich. Es ist zudem möglich den piezoelektrischer Kristall als ein Empfänger zu nutzen.







