\section{Diskussion}
\label{sec:Diskussion}
Zuerst werden die Positionen der Störstellen die mit dem IEV bestimmt wurden (Tabelle \ref{tab:1}) 
mit den mit der Schiebelehre aufgenommenen Messwerten (Tabelle \ref{tab:2}) verglichen.
In Tabelle \ref{tab:disk} befinden sich die relativen Abweichungen der Positionen der einzelnen Störstellen.


\begin{table}
    \centering
    \caption{Relative Abweichungen der Störungsvermessung.}
    \begin{tabular}{c c}
        \toprule
        {Nr.} & {$\Delta_{rel} d \, / \, \%$} \\
        \midrule
     1  & 1.7  \\
     2  & 0.3 \\
     3  & 6.2 \\
     4  & 5.5  \\
     5  & 2.9  \\
     6  & 3.1 \\
     7  & 2.7  \\
     8  & 1.8  \\
     9  & 1.7  \\    
     10 &       \\
     11 & 3.7 \\
        \bottomrule
    \end{tabular}
    \label{tab:disk}
\end{table}

\noindent
Ein Nachteil am IEV ist demnach, dass Störstellen von anderen Störstellen verdeckt werden können und so unbemerkt bleiben.
Im Allgemeinen treten jedoch nur kleine Abweichungen auf, die im Rahmen der Messungenauigkeiten liegen.
Daher kann die Vermessung der Störstellen durch das IEV als erfolgreich angesehen werden.

\noindent
Wird die ermittelte Schallgeschwindigkeit $c_{Acryl} = 2732.2 \si[per-mode=fraction]{\meter\per\second}$ 
mit dem Literaturwert $c_{lit} = 2730 \si[per-mode=fraction]{\meter\per\second}$ verglichen, 
ergibt sich eine relative Abweichung von $\Delta_{rel} c_{Acryl} = 0.3 \%$.
Somit konnte die Schallgeschwindigkeit in Acryl mit großer Genauigkeit experimentell bestimmt werden.
Dabei ist zu erwähnen dass sich die Literaturangaben unterschiedlicher Quellen leicht voneinander unterscheiden.

\noindent
Zur Dämpfungskonstante $\alpha$ und zum Augenmodell liegen keine Literatur- oder Referenzwerte vor, 
weswegen der Teil des Versuchs nicht diskutierbar ist.